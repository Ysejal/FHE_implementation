\begin{section}{Introduction}
Certains systèmes cryptographiques, comme RSA, ElGamal ou encore le cryptosystème de Pallier possèdent une
propriété intéressante : faire le produit de deux chiffrés revient à chiffrer le produit de leurs clairs. Cette
propriété offre à un attaquant des informations qui affaiblissent le chiffré - on parle de malléabilité -, mais
offre aussi des perspectives intéressante : la possibilité d'appliquer des opérations sur les chiffrés de
données sans avoir à les déchiffrer permet de travailler avec des entités sans avoir leur dévoiler des données que l'on estime sensibles.
	\paragraph{}
	Toutefois, pouvoir uniquement multiplier des chiffrés\footnote{ou bien seulement les sommer, comme cela peut
	être le cas, par exemple avec des variantes de ElGamal} est trop limité, ne permettant pas de faire beaucoup de manipulations intéressantes. Il a fallu attendre jusqu'en 2009, avec la thèse de Craig Gentry \cite{gentry_thesis} pour que devienne plausible la possibilité d'un cryptosystème \og commutant \fg~ à la fois avec la somme, le produit et la multiplication par des scalaires; on parle alors de \og Fully homomorphic cryptosystem \fg~ (FHE).

	\paragraph{}
	Celui-ci utilisait des réseaux euclidiens dits \og idéaux\fg~ ainsi qu'une technique dite de \og bootstrapping \fg~ pour passer d'un \og Somewhat fully homomorphic cryptosystem \fg, limitant le nombre d'opérations possibles afin que le déchiffrement fonctionne toujours, vers un vrai FHE.

A partir de là, de nombreuses tentatives de FHE ont émergé, basant essentiellement leur sécurité sur le problème
dit du Learning With Error (LWE) mis en avant en 2005 par Oded Regev dans l'article \cite{STOC:Regev05}.
En font partie les cryptosystèmes dits de seconde génération apparus vers 2011. 
Il furent les premiers à permettre une implémentation \og réaliste \fg, et avaient la particularité d'avoir une somme entre chiffrés facile à
mettre en place, tandis que la multiplication demandait elle plus de travail, demandant notamment une opération
dite de \og relinéarisation \fg~ assez complexe.  Une troisième génération est ensuite apparue avec le cryptosystème
GWS, publié  par Gentry, Sahai et Waters en 2013 (voir \cite{EPRINT:GenSahWat13}), se distinguant par 
une nouvelle approche permettant une définition du produit aussi simple que la somme.

Le but de notre rapport et des codes associés, présents dans le github CITERNOTREGITHUB, est d'étudier 
le cryptosystème GSW à la fois sous des aspects théoriques et pratiques. 
Pour cela, nous le présenterons, étudierons sa sécurité et les paramètres permettant de l'assurer, puis en présenterons une implémentation jouet
\footnote{sans regards sur les très nombreuses optimisations aujourd'hui faites dans les vraies implémentations
de GSW et ses dérivés} faite en \path{sagemath} que nous avons réalisée pour ce projet.
Enfin, nous jetterons aussi un \oe{}il à des API open-source permettant d'utiliser des FHE.
Afin d'être exhaustif, nous avons essayé de donner des définitions pour chacune des notions évoquées dans notre travail,
notamment concernant les gaussienne discrètes et les diverses définitions associées aux FHE. Notez toutefois que 
celles-ci sont généralement tirés, ou inspirées d'autres expositions plus détaillées, qui seront évidemment précisés
et que nous conseillons au lecteur de se référer. En ce sens, ce rapport permet aussi de faire une petite synthèse 
bibliographique.
\end{section}
