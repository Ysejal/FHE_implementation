\documentclass{article}

\usepackage{ifxetex}

  \usepackage[T1]{fontenc}
  \usepackage[utf8]{inputenc}
  \usepackage[frenchb]{babel}
  \usepackage{amsmath,amsfonts,amssymb,amsthm}
  \usepackage{url}
  \usepackage{graphicx}
  \usepackage{listings}
  \usepackage[left=3cm, right=3cm, top=2cm, bottom=2cm]{geometry}

  \usepackage{tikz}
  \usetikzlibrary{graphs, graphs.standard, quotes}% quotes library is for the [""] edges

  \newtheorem{definition}{Définition}
  \newtheorem{thm}{Théorème}
  \newtheorem{lemme}{Lemme}
  \newtheorem{prop}{Proposition}


  % for double brackets \llbracket \rrbracket
  \usepackage{stmaryrd}

  % for the "say" command, to make quotes
  \usepackage{csquotes}

  \usepackage{caption}

  \usepackage{color}
  \definecolor{dblackcolor}{rgb}{0.0,0.0,0.0}
  \definecolor{dbluecolor}{rgb}{0.01,0.02,0.7}
  \definecolor{dgreencolor}{rgb}{0.2,0.4,0.0}
  \definecolor{dgraycolor}{rgb}{0.30,0.3,0.30}
  
  % used defined commands
  \newcommand{\LWE}{\text{LWE}_{n,q,\chi}}
  \newcommand{\DLWE}{\text{DLWE}_{n,q,\chi}}
  \newcommand{\SD}{\text{SD}}
  \newcommand{\id}{\text{I}}
  \newcommand{\flatten}{\text{Flatten}}
  \newcommand{\bitdecomp}{\text{BitDecomp}}
  \newcommand{\ZZ}{\mathbb{Z}}
  \newcommand{\NN}{\mathbb{N}}
  \newcommand{\RR}{\mathbb{R}}
  \newcommand{\ZZq}{\mathbb{Z}_q}
  \newcommand{\EE}{\mathbb{E}}
  \newcommand{\PP}{\mathbb{P}}

%%%%%%%%%%%%%%%%%%%%%%%%%%%%%%%%%%%%%%%
% the basic environement, for Ocaml
% to use: just \begin{lstlisting}
%%%%%%%%%%%%%%%%%%%%%%%%%%%%%%%%%%%%%%%

  \lstset{
          mathescape=true,
  	  frame=none,
	  showtabs=False,
	  showspaces=False,
	  showstringspaces=False,
	  commentstyle={\ttfamily\color{dgreencolor}},
	  keywordstyle={\ttfamily\color{dbluecolor}\bfseries},
	  stringstyle={\ttfamily\color{dgraycolor}\bfseries},
	  language=python, % TO CHANGE
	  basicstyle={\fontsize{10pt}{10pt}\ttfamily},
	  aboveskip=0.3em,
	  belowskip=0.1em,
	  numbers=none,
	  numberstyle=\footnotesize,
	  inputpath=codes
	  % to have caption under the code 
  }



%%%%%%%%%%%%%%%%%%%%%%%%%%%%%%%%%%%%%%%
% the algorithm environement, for pseudo-code
% to use: just \begin{algorithm}[caption={CAPTION}, label={LABEL}]
%%%%%%%%%%%%%%%%%%%%%%%%%%%%%%%%%%%%%%%

% this environment has been found in:
% https://tex.stackexchange.com/questions/111116/
% what-is-the-best-looking-pseudo-code-package

% u{lg}[chapter] % defines algorithm counter for chapter-level

%defines appearance of the algorithm counter
\newcounter{nalg}[section] % defines algorithm counter for chapter-level
\renewcommand{\thenalg}{\thechapter .\arabic{nalg}} 
% defines a new caption label as Algorithm x.y
\DeclareCaptionLabelFormat{algocaption}{Algorithm \thenalg} 

\lstnewenvironment{algorithm}[1][] %defines the algorithm listing environment
{   
	%increments algorithm number
    \refstepcounter{nalg}
	%defines the caption setup for: 
    \captionsetup{labelformat=algocaption,labelsep=colon} 
    % it ises label format as the declared caption label above and makes label and caption 
    % text to be separated by a ':'
    \lstset{ %this is the stype
        mathescape=true,
        frame=tB,
        numbers=left, 
        numberstyle=\tiny,
        basicstyle=\scriptsize, 
        keywordstyle=\color{black}\bfseries\em,
        %add the keywords you want, or load a language 
        % as Rubens explains in his comment above.
        keywords={,Input, Output, return, datatype, function, in, if, else, 
                   foreach, while, begin, end, do} 
        numbers=left,
        xleftmargin=.04\textwidth,
        #1 % this is to add specific settings to an usage of this environment 
        % (for instance, the caption and referable label)
    }
}
{}


\title{Implémentation d'un FHE}
\author{Lucas Roux  et Eric Sageloli}

\graphicspath{{pictures/}}
% to make an entry for the bibliography in the table of contents
\usepackage[nottoc, notlof, notlot]{tocbibind}

\begin{document}
\maketitle
\newpage
\tableofcontents{}
\newpage
\begin{section}{Notations}
\end{section}

\newpage
\lstlistoflistings
\begin{section}{Introduction}
	\begin{lstlisting}[caption=ceci est un test, captionpos=b]
for moi in range(haha):
	je suis un oiseau
\end{lstlisting}
\end{section}

\newpage
\begin{section}{Notions préliminaires}
	\begin{subsection}{LWE et DLWE}
	Nous présentons ici les définitions du Learning with Error (LWE) dans leur version décisionnelle et calculatoire.
	
	\begin{definition}{Decisional Learning with Errors (DLWE), version décisionnelle}

	Pour un paramètre de sécurité $\lambda$, soit $n = n(\lambda),\ q = q(\lambda)$ des entiers et $\chi = \chi(\lambda)$ une distribution sur $\ZZ$.
	
	Le problème $\DLWE$ consiste à devoir distinguer deux distributions sur $\ZZq^{n+1}$ à partir d'un nombre polynomial $m = m(\lambda)$ d'échantillons qu'une des deux à produite. La première distribution crée des vecteurs $(\vec{a}_i,b_i) \in \ZZq^{n+1}$ uniforme. La deuxième utilise un $\vec{s} \in \ZZq^n$ tiré uniformément et prend pour valeurs des vecteurs $(\vec{a}_i, b_i)$ pour lesquels :
	\[ b_i = \langle \vec{a}_i, \vec{s} \rangle + e_i \]
	où $e_i$ est crée obtenue par $\chi$. \\
	\end{definition}
	Notons qu'alors, $n = O(P(\lambda))\text{ et }\log(q) = O(P(\lambda))$ pour un polynome $P$.

	\begin{definition}{Learning with Errors (LWE)}
	
	Pour un paramètre de sécurité $\lambda$, soit $n = n(\lambda),q = q(\lambda)$ des entiers et $\chi = \chi(\lambda)$ une distribution sur $\ZZ$. On tire $\vec{s} \in \ZZq^n$ uniformément et on considère la distribution qui prend pour valeurs des vecteurs $(\vec{a}_i, b_i)$ pour lesquels :
	\[ b_i = \langle \vec{a}_i, \vec{s} \rangle + e_i \]
	où $e_i$ est crée obtenue par $\chi$.

	Le problème $\LWE$ consiste à trouver $\vec{s}$ à partir d'un nombre polynomial $m = m(\lambda)$ d'échantillons.
	\end{definition}

	Ces deux problèmes sont en fait \og équivalents \fg. C'est assez évident de LWE vers DLWE. De plus, le Lemme 4.2 de \cite{STOC:Regev05} montre comment réduire à DLWE à LWE sous certaines hypothèses lorsque $q$ est premier et $q = \mathcal{O}(\text{poly}(n))$. Le théorème 3.1 de \cite{EPRINT:MicPei11} montre la même chose mais lorsque $q$ est un produit de premiers $p_i \in \mathcal{O}(\text{poly}(n))$, comme ce sera le cas lorsque nous considèrerons $q = 2^k$.

	Voyons par exemple le cas (plus facile) où $q$ est premier :

	\begin{prop}{DWLE vers LWE}

	Soit $n \geqslant 1$ un entier, $2 \leqslant q \leqslant \text{poly}(n)$ un nombre premier et $\chi$ une distribution sur $\ZZq$. Supposons avoir accès à un automate $\mathcal{W}$ qui accepte avec une probabilité exponentiellement proche de 1 les distributions $A_{s, \xi}$ et rejete avec une probabilité exponentiellement proche de 1 la distributions uniforme $U$.
	
	Il existe alors un automate $\mathcal{V}$ qui, étant donné des échantillons de $\mathcal{A}_{s,\chi}$ pour un certain $s$, retrouve $s$ avec une probabilité exponentiellement proche de 1.
	\end{prop}
	\begin{proof}
	Nous indiquons ici la démontration faite dans \cite{STOC:Regev05}.
	
	L'automate $W'$ va trouver $s$ coordonée par coordonnée. Montrons comment $W'$ obtient la première coordonnée $s_1$.
	
	Pour $k \in \ZZq$, on considère la fonction :
	\[f_{k,1}: (a,b) \mapsto (a + (l, 0, ..., 0), b + l \cdot k) \]
	avec $l\in \ZZq$ échantilloné uniformément sur $\ZZq$.
	
	$f_{k,1}$ appliquée à un échantillon uniforme donne un échantillon uniforme tandis qu'appliquée à un échantillon de $A_{s, \chi}$, elle donne un échantillon de $A_{s, \chi}$ si $k = s_1$, et uniforme sinon.
	
	On peut faire une recherche exhaustive sur les $k \in \ZZq$ jusqu'à en trouver un accepté par $W$, qui sera le bon avec une probabilité exponentiellement proche de 1.
	
	Cela se fait en temps polynomial car $q < \text{poly}(n)$ et $f_{k,1}$ s'execute en temps polynomial.
	
	On peut effectuer la même chose avec la fonction
	\[f_{k,i}: (a,b) \mapsto (a + (0, 0, ..., l, 0, ..., 0), b + l \cdot k) \]
	avec le $l$ ajouté à $a$ en $i$ème position $\forall i$.
	
	On retrouve ainsi $s$ avec $n$ calculs polynomiaux en $n$, ce qui reste évidemment polynomial en $n$.
	
	La probabilité de se tromper est $n$ fois quelque chose d'exponentiellement proche de 0 et reste donc exponentiellement proche de 0.
	\end{proof}

	Pour analyser la sécurité du cryptosystème, nous utiliserons le problème DLWE. Comme l'indique le théorème 1 de \cite{C:GenSahWat13}, il est possible de réduire le problème LWE à des problèmes sur des réseaux.

	Indiquons ici de façon informelle comment passer du problème LWE à un problème de type SVP (short vector problem). Tout d'abord, nous aurons besoin d'exprimer LWE sous une forme matricielle :

	\begin{definition}{versions matricielles de DLWE et LWE}

	En prenant les paramètres de la précédente définition, le problème $\DLWE$ consiste à décider si une matrice $A \in \ZZq^{m \times (n+1)}$ est uniforme ou bien s'il existe un vecteur $\vec{v} = (1\quad -\vec{s})$ tel que $A \cdot \vec{v} \in \ZZq^{m}$ est créé à partir de $\chi^m$. Autrement dit, avec les notations de la formulation classique de LWE, si les lignes de $A$ sont de la forme $(b_i, \vec{a}_i)$.
	
	Le problème $\LWE$ consiste lui à trouver $\vec{v}$ à partir de $A$.
	\end{definition}

	Nous allons ici considérer le problème LWE calculatoire, dans lequel il faut trouver le vecteur $\vec{v}$ tel que :
	\[ A\cdot \vec{v} = \vec{e} \mod q \]
	où les coordonnées de $\vec{e}$ sont créées par $\chi$.

	De façon équivalente, il faut trouver un vecteur $(*\quad\vec{v})$ tel que :
	\[ \begin{bmatrix}q & A \\ 0 &1 \end{bmatrix}\cdot
	   \begin{bmatrix}* \\ \vec{v} \end{bmatrix} =
	   \begin{bmatrix} \vec{e} \\ \vec{v} \end{bmatrix} \]
	Si la distribution $\chi$ créé de petites valeurs, on voit qu'on a alors trouvé un \og petit \fg vecteur du réseau engendré par les colonnes de 
	\[ \begin{bmatrix}q & A \\ 0 &1 \end{bmatrix} \]
	\end{subsection}

	\begin{subsection}{Réseaux euclidiens}
	Nous rappelons ici quelques résultats sur les réseaux euclidiens, tels qu'énoncés dans \cite{EC:MicPei12}. Ils nous serons utiles pour définir les gaussiennes discrètes ainsi que pour comprendre un des algorithme de déchiffrement du cryptosystème GSW.

	Tous les réseaux considérés ici sont de rang plein, autrement dit, si $\L \subset \RR^n$, alors $L$ est de dimension $n$.

	\begin{definition}
	Soit $L\subset \RR^n$ un réseau. Le dual de $L$ est défini comme étant :
	\[ L^* = \{ v \in \RR^n \::\: \langle x, v \rangle \in \ZZ
	   \:\text{ pour tout } x\in L\} \]
	\end{definition}
	\begin{prop} 
	Soit $L \subset \RR^n$ un réseau euclidien. Soit $B$ est une base de $L$.
	
	$B^{-t}$ est une base de $L^*$.
	\end{prop}

	Pour $q$ un entier et $A \in \ZZ^{n\times m}$, on pose :
		\[\Lambda^\bot(A) = \{ z \in \ZZ^m\: :\: A z  = 0 \mod q \}\] 
	\[\Lambda(A^t) = \{ z \in \ZZ^m\: : \: \exists s\in \ZZq^n, z = A^t s \mod q\}\]

		\begin{prop} \label{lambda_reseau}
	Conservant les notations précédentes, 
	\[q \cdot {\Lambda^\bot(A)}^* =  \Lambda(A^t)\] 
	\end{prop}
	\end{subsection}
	
	\begin{subsection}{La gaussienne discrète}
	Très souvent, la distribution $\chi$ choisi pour avoir des paramètres sécurités pour le problème LWE est une gaussienne discrète. Nous nous proposons ici d'en indiquer la définition, ainsi que d'en indiquer certaines propriétés.

	Rappellons aussi qu'une famille $\{\chi_n\}_n$ de distributions est dite $B$-bornée pour une borne $B = B(n)$ si la fonction suivante est négligeable :
		\[n \mapsto \PP\left(\chi_n > B(n)\right) \]

	\paragraph{}
	Nous reprenons ici les notations de \cite{STOC:GenPeiVai08}.

	Soit un entier $n > 0$  et $\sigma > 0$. On définit la densité gaussienne sur $\RR^n$ comme la fonction qui à $x\in\RR^n$ attribue:
	\[\rho_{s,c}(x) = e^{\phi * {\frac{||x-c||}{s}}^2} \]

	Puis, pour un réseau $\Lambda \in \RR^n$, nous définissons la gaussienne discrète $D_\alpha$ comme la distribution de support $\Lambda$ de loi de probabilité: 	
	\[ D_{\Lambda, s, c}(x) = \frac{\rho_{s,c}(x)}{\sum_{l\in \Lambda}\rho_{s,c}(l)}\]

	Enfin, pour un entier $q > 0$, nous définissons la gaussienne discrète $D^q_{\alpha}$ modulo un entier $q > 0$ comme la composition la fonction qui a $x \in \ZZq$ attribue 
	
	LA COMPOSITION LA FONCTION ? WTF ?  
		\[ D_{\ZZ, s, c}(\pi^{-1}(x)) \]
	où $\pi$ est la projection $\ZZ \rightarrow \ZZq$.

La définition et la position suivante (Le lemme 4.2 de \cite{STOC:GenPeiVai08}) permettent de trouver une borne à une famille de gaussienne.
\begin{definition}
Pour un réseau $L$ de dimension $n$ et un réel $\epsilon > 0$, le paramètre $\eta_\epsilon(L)$ est le plus petit réel $s>0$ tel que 
	\[\rho_{1/s}(L^* \setminus \{0\}) \leqslant \epsilon\]
\end{definition}
	\begin{prop}
	Pour tout $\epsilon > 0$ , $s \geqslant \eta_{\epsilon}(\ZZ)$ et pour tout $t>0$ :
	\[ \PP\left(|x-c| \geq t\cdot s\right) \leqslant 2 e^{-\pi t^2}	\cdot \frac{1+\epsilon}{1-\epsilon} \]
	Notamment, pour $0 < \epsilon  < 1/2$ et $t \geqslant \omega(\sqrt{\log(n)})$, cette probabilité est négligeable.
	\end{prop}
	\end{subsection} % gaussiene discrete
\end{section}

\begin{section}{Présentation du cryptosystème}
	\begin{subsection}{L'idée générale}
	L'idée de ce cryptosystme consiste à prendre pour secret un
	certain vecteur $\vec{v} \in \ZZq^N$ pour certains paramètres
	$q, N \in \NN$, puis à chiffrer un message $m \in \ZZq$ à
	l'aide d'une matrice $C \in \ZZq^{N \times N}$ ayant $m$ pour valeur propre 
	associée au vecteur propre $\vec{v}$. Autrement dit, avec:
	\[C \cdot \vec{v} = m \vec{v}\: \mod q \]

	De là, il est facile de voir que pour $\lambda \in \ZZ$ et  $C_1$ et $C_2$ 
	chiffrés de $m_1$ et $m_2$, on a :
	\begin{align*}
	& (C_1 + C_2) \cdot \vec{v} = (m_1 + m_2) \vec{v} \\
	& (C_1 \times C_2) \cdot \vec{v} = (m_1 + m_2) \vec{v} \\
	& (\lambda  C_2) \cdot \vec{v} = (\lambda m_1) \vec{v} 
	\end{align*}

	Toutefois, un tel système n'est pas sécurisé car $C$ n'a qu'un nombre
	fini de valeurs propres, et il semble donc facile de 
	retrouver le secret $\vec{v}$.

	La solution consiste alors à ajouter du bruit au chiffré, c'est à dire
	à chiffrer $m\in \ZZq$ par une matrice $C \in \ZZ^{N \times N}$
	telle que:
	\[ C \vec{v} = m \vec{v} + \vec{e} \]
	pour une \og petite \fg erreur $\vec{e}$. Si le vecteur
	$\vec{v}$ contient un grand coefficient $v_i$, on voit alors qu'il 
	reste possible de retrouver $m$ avec
	\begin{align*}
	\frac{{(C \vec{v})}_i}{v_i} = \frac{m + e_i}{v_i}
	\end{align*}
	
		Nous verrons que pour de bons
	choix de paramètres, déchiffrer un tel message permet de
	résoudre une instance de LWE.

	Toutefois, l'ajout d'une erreur comporte ses inconvénients. Si nous
	revenons aux équations précédentes, en introduisant les erreurs 
	$\vec{e}_i$ pour chiffrer  $m_i$ ($i\in \{1,2\}$), on obtient :

	\begin{align*}
	& (C_1 + C_2) \cdot \vec{v} = (m_1 + m_2) \vec{v} + (\vec{e}_1 + \vec{e}_2)\\
		& (C_1 \times C_2) \cdot \vec{v} = (m_1 * m_2) \vec{v} + C_1
		\vec{e}_2 + m_2\vec{e}_1 \\
	& (\lambda  C_2) \cdot \vec{v} = (\lambda m_1) + \lambda e_i\vec{v} 
	\end{align*}

	Notamment, on voit que le terme $C_1 * \vec{e}_2$ peut être très grand 
	même pour un petit $\vec{e}_2$. Nous verrons par la suite comment
	choisir nos paramètres, et notamment $\vec{v}$, afin de 
	toujours pouvoir se ramener à des chiffrés $C \in \{0,1\}^{N
	\times N}$. De cette façon, on aura:

	
	
	




		% Explication sans l'erreur, pourquoi c'est homomophique
	\end{subsection}
	\begin{subsection}{Fonctions utilisées}
	Plusieurs algorithmes seront utiles pour pouvoir bien définir le système FHE :
	
	\paragraph{}
	\textbf{BitDecomp}
	\flushleft

	\textbf{Entrée} : Cet algorithme prends en entrée un vecteur $\vec{a} = (a_1, ..., a_k) \in \ZZq^{k}$. \\
	\textbf{Sortie} : Cet algorithme retourne la décomposition binaire des éléments de $\vec{a}$ sous la forme d'un vecteur. \\
	\textbf{Algorithme} : Pour chaque $a_i$, on détermine sa représentation binaire avec les bits de faibles puissance à gauche et non à droite. On retourne la concaténation de ces représentations binaires sous la forme d'un vecteur.
	
	\paragraph{}
	\textbf{BitDecomp}$^{-1}$
	\flushleft

	\textbf{Entrée} : Cet algorithme prends en entrée un vecteur $\vec{a} = (a_{1,0}, ..., a_{1,l-1}, a_{2,0}, ..., a_{k,l-1})$. \\
	\textbf{Sortie} : Cet algorithme renvoie ($\sum\limits_{i=0}^{l-1} 2^{i} a_{1,i}, ..., \sum\limits_{i=0}^{l-1} 2^{i} a_{k,i})$. \\
	\textbf{Remarque} : Si tous les $a_{i,j}$ sont dans $\{ 0,1 \} $, cet algorithme inverse bien \textbf{BitDecomp}, cependant, sa définition ne le limite pas aux vecteurs $\in \{ 0,1\} ^{k\times l}$.

	\paragraph{}
	\textbf{Flatten}
	\flushleft

	\textbf{Entrée} : Cet algorithme prends en entrée un vecteur $\vec{a} = (a_{1,0}, ..., a_{1,l-1}, a_{2,0}, ..., a_{k,l-1})$. \\
	\textbf{Sortie} : Cet algorithme retourne un vecteur $\vec{b} = (b_{1,0}, ..., b_{1,l-1}, b_{2,0}, ..., b_{k,l-1})$ dont les éléments sont tous dans $\{ 0,1\} $. \\
	\textbf{Algorithme} : On calcule \textbf{BitDecomp}$^{-1}(\vec{a})$ et on obtient un vecteur $\vec{z} \in \ZZq^{k}$. On applique ensuite \textbf{BitDecomp} à $\vec{z}$ et l'on renvoie le résultat obtenu.
	
	\paragraph{}
	\textbf{PowersOf2}
	\flushleft

	\textbf{Entrée} : Cet algorithme prends en entrée un vecteur $\vec{a} = (a_1, ..., a_k) \in \ZZq^{k}$. \\
	\textbf{Sortie} : Cet algorithme renvoie ($a_1, 2 a_1, 2^{2} a_1, ..., 2^{l-1} a_1, a_2, ..., 2^{l-1} a_k)$. \\
	
	\begin{prop}
	Soient $\vec{a}$ et $\vec{b}$ dans $\ZZq^{k}$. \\
	On a $\langle \textbf{BitDecomp}(\vec{a}), \textbf{PowersOf2}(\vec{b}) \rangle = \langle\vec{a},\vec{b} \rangle$.
	\end{prop}
	\begin{proof}
	\begin{align*}
	\langle \textbf{BitDecomp}(\vec{a}), \textbf{PowersOf2}(\vec{b}) \rangle &= \sum\limits_{i=1}^{k} \sum\limits_{j=0}^{l-1} a_{i,j} * (2^{j} * b_i) \\
	&= \sum\limits_{i=1}^{k} b_i * \sum\limits_{j=0}^{l-1} (a_{i,j} * 2^{j}) \\
	&= \sum\limits_{i=1}^{k} b_i * a_i \\
	&= \langle\vec{a},\vec{b} \rangle.
	\end{align*}
	\end{proof}
	
	\begin{prop}
	Soient $\vec{a}$ dans $\ZZq^{k \times l}$ et $\vec{b}$ dans $\ZZq^{k}$. \\
	On a $\langle \vec{a}, \textbf{PowersOf2}(\vec{b}) \rangle = \langle \textbf{BitDecomp}^{-1}(\vec{a}), \vec{b}\rangle = \langle \textbf{Flatten}(\vec{a}),\textbf{PowersOf2}(\vec{b}) \rangle$. \\
	\end{prop}
	\begin{proof}
	\begin{align*}
	\langle \vec{a}, \textbf{PowersOf2}(\vec{b}) \rangle &= \sum\limits_{i=1}^{k} \sum\limits_{j=0}^{l-1} a_{j+li} * (2^{j} * b_i) \\
	&= \sum\limits_{i=1}^{k} b_i * \sum\limits_{j=0}^{l-1} (a_{j+li} * 2^{j}) \\
	&= \langle \textbf{BitDecomp}^{-1}(\vec{a}), \vec{b}\rangle \\
	\end{align*}
	Soit $c = \textbf{BitDecomp}^{-1}(\vec{a})$.
	\begin{align*}
	\langle \textbf{Flatten}(\vec{a}),\textbf{PowersOf2}(\vec{b}) \rangle &= \langle \textbf{BitDecomp}(\vec{c}),\textbf{PowersOf2}(\vec{b}) \rangle \\
	&= \sum\limits_{i=1}^{k} \sum\limits_{j=0}^{l-1} c_{i,j} * (2^{j} * b_i) \\
	&= \sum\limits_{i=1}^{k} b_i * \sum\limits_{j=0}^{l-1} (c_{i,j} * 2^{j}) \\
	&= \sum\limits_{i=1}^{k} b_i * c_i \\
	&= \langle \textbf{BitDecomp}^{-1}(\vec{a}), \vec{b}\rangle \\
	&= \langle \vec{a}, \textbf{PowersOf2}(\vec{b}) \rangle
	\end{align*}
	\end{proof}
	
	\end{subsection}
	\begin{subsection}{Définition du cryptosystème}
	On rappelle que les paramètres du système défini ici sont : le paramètre de dimension $n$, le modulus $q$, un modèle de distribution de l'erreur $\chi$ ainsi que $m$, qui, tout comme $n$ influera la taille des matrices manipulées. \\
	On note $l = \lfloor$log $q\rfloor + 1$ et $N = (n + 1)$ $l$.
		
	\paragraph{}
	\textbf{Setup}
	\flushleft

	\textbf{Entrée} : Cet algorithme prends en entrée $1^\lambda$ et $1^L$ avec $\lambda$ paramètre de sécurité et L paramètre de profondeur. \\
	\textbf{Sortie} : Cet algorithme retourne les paramètres $n, q, \chi, m$ du système. \\
	\textbf{Algorithme} : On définit des paramètres permettant de pouvoir effectuer au moins L opérations sur un chiffré et de toujours pouvoir le déchiffrer correctemment tout en assurant qu'un adversaire attaquant le système doive effectuer au moins $2^\lambda$ opérations, quelle que soit l'attaque qu'il choisisse. La façon de déterminer ces paramètres n'est pas définie afin de pouvoir l'adapter suivant l'evolution des attaques. \\
	
	\paragraph{}
	\textbf{SecretKeyGen}
	\flushleft
	
	\textbf{Entrée} : Cet algorithme n'a besoin en entrée que des paramètres donnés par \textbf{Setup}. \\
	\textbf{Sortie} : Cet algorithme retourne la clé secrète $\vec{s} \in \ZZq^{n+1}$. \\
	\textbf{Algorithme} : On génère aléatoirement un vecteur $\vec{t} \in \ZZq^n$. On définit la clé secrète comme $\vec{s} = (1, -t_1, ..., -t_n)$. \\
	\textbf{Taille} : Comme on l'a dit, $\vec{s} \in \ZZq^{n+1}$. Par définition, $q$, et donc tous élément de $\ZZq$, s'écrit en $l$ bits. $\vec{s}$ fait donc une taille de $l * (n+1) = N$ bits. \\
	On note $\vec{v} = $ \textbf{PowersOf2}$(\vec{s})$.
	
	\paragraph{}
	\textbf{PublicKeyGen}
	\flushleft
	
	\textbf{Entrée} : Cet algorithme n'a besoin en entrée que des paramètres donnés par \textbf{Setup} et d'une clé secrète construite avec ces mêmes paramètres. \\
	\textbf{Sortie} : Cet algorithme retourne la clé publique $A \in \ZZq^{m \times n}$.\\
	\textbf{Algorithme} : On génère une matrice uniforme $B \in \ZZq^{n \times m}$ et un vecteur $\vec{e}$ de m éléments choisis suivant la distribution $\chi$. On définit $\vec{b} = B \times \vec{t} + \vec{e}$. La clé publique est la matrice constituée de l'indentation de $\vec{b}$ considéré comme un vecteur colonne et de $B$. \\
	\textbf{Taille} : $A \in \ZZq^{m \times n}$ donc A s'écrit en $l * n * m = N * (m - 1)$ bits.
	
	\paragraph{}
	\textbf{Encrypt}
	\flushleft
	
	\textbf{Entrée} : Cet algorithme prend en entrée les paramètres du système, la clé publique et un message $\mu \in \ZZq$. \\
	\textbf{Sortie} : Cet algorithme retourne le chiffré $C \in \ZZq^{N \times N}$ de $\mu$.\\
	\textbf{Algorithme} : On génère uniformément une matrice $R \in \{ 0,1\} ^{N \times m}$. Le chiffré est : $C = $ \textbf{Flatten}$(\mu \times I_N + $\textbf{BitDecomp}$(R \times A))$. \\
	\textbf{Taille} : $C \in \ZZq^{N \times N}$ s'écrit en $l * N^2$ bits.
	
	\paragraph{}
	\textbf{Dec}
	\flushleft
	
	\textbf{Entrée} : Cet algorithme prend en entrée les paramètres du système, la clé secrète et un chiffré d'un message $\mu \in \{ 0,1\} $. \\
	\textbf{Sortie} : Cet algorithme retourne le clair du chiffré si l'erreur de ce dernier n'est pas trop élevée.\\
	\textbf{Algorithme} : On rappelle que les $l$ premiers coefficients de $\vec{v}$ sont les puissances de 0 à $l-1$ de 2. Soit $i \leqslant l$ tel que le i+1ème coefficient de $\vec{v}$, égal à $2^{i}$, soit compris entre $q/4$ et $q/2$, $q/2$ compris. On note $C_i$ la ième ligne de $C$. On calcule ensuite $x_i = \langle C_i, \vec{v} \rangle$ et on renvoie $\lfloor x_i/v_i \rceil$.


\paragraph{}
\begin{definition}
On appellera erreur d'un chiffré $C$ d'un message $\mu$ le vecteur $\vec{e}$ tel que  
\[ C\cdot \vec{v} = \mu\, \vec{v} + \vec{e} \]
\end{definition}
\begin{prop}
	\textbf{Dec} décrypte avec succès les chiffrés dont l'erreur $\vec{e}$ satisfait $\|\vec{e}\|_1 < q/8$.
\end{prop}
\begin{proof}
	On a $x_i = \mu * v_i + e$ avec $\lvert e \lvert \leqslant \|\vec{e}\|_1$ et $\frac{x_i}{v_i} = \mu + \frac{e}{v_i}$. \\
	$\lvert v_i \lvert > \frac{q}{4}$, d'où $\lvert \frac{e}{v_i} \lvert < 1/2$, donc $\lfloor \frac{x_i}{v_i} \rceil = \mu$.
\end{proof}
	\end{subsection}
	
\begin{subsection}{Autres algorithmes de déchiffrement}
L'algorithme de déchiffrement que nous avons présenté fonctionne sans
contraintes sur $q$ mais ne déchiffre que des chiffrés de $0$ et de $1$.
	
Nous proposons ici une analyse un peu plus fine du déchiffrement pour
montrer comment faire pour des chiffrés de n'importe quel élément de $\ZZq$.

Pour cela, remarquons qu'en partant d'un chiffré  $C$
de $m \in \ZZq$
On a: 
\[ C \cdot \vec{v} = m \vec{v} + \vec{e} \mod q \]
pour une erreur $\vec{e}$. \\
En considérant l'équation sur les $l$ premières coordoonées, on obtient:

	\[\vec{a} = m \vec{p} + \vec{e} \mod q\quad \text{où}\quad \vec{p} = (1\:2\:\cdots\:2^{l-1}) \]
 
Notant $L = \Lambda(\vec{p}^t)$, on constate qu'on peut 
retrouver $m\vec{p}$ en trouvant le vecteur de $L$
le plus proche de $\vec{a}$

De cette idée, on déduit 2 algorithmes de déchiffrements 
supplémentaires, dépendant de la façon dont 
on résout le problème du vecteur le plus proche.


\begin{itemize}
\item \textbf{mp\_decrypt} qui suppose que $q$ est 
	une puissance de $2$.
\item \textbf{mp\_all\_q\_decrypt} sans hypothèses sur $q$.
\end{itemize}

\begin{subsubsection}{mp\_decrypt: $q$ est une puissance de $2$}
L'algorithme, présenté dans  \cite{EPRINT:GenSahWat13}, utilise 
le fait que $q = 2^l$.

En regardant la dernière coordonnée de:
\[\vec{a} = m \vec{p} + \vec{e} \mod q\quad \text{où}\quad \vec{p} = (1\:2\:\cdots\:2^{l-1}) \]
 
On obtient:
\[m 2^{l-1} + e_{l} \mod 2^l \]
qui est proche de $0$ si $m$ est pair et de $q/2$ sinon. On déduit de cette 
façon le premier bit de l'écriture en binaire de $m$ et la méthode est similaire
pour complètement déduire $m$.
	
\end{subsubsection}
\begin{subsubsection}{mp\_all\_q\_decrypt: $q$ est quelconque}
Le travail effectué ici est notamment tiré de la section 4 de
\cite{EC:MicPei12}.

En utilisant la proposition~\ref{lambda_reseau}, on constate que
\[ L = q \cdot \Lambda^\bot\left(\vec{p}\right) \]
Il nous suffit donc de trouver une base $B$ de $\Lambda^\bot\left(\vec{p}\right)$
pour en déduire une base $q \, B^{-t}$ de $L$.

Or, il est facile de voir que  
\[ B = \begin{bmatrix} 
	2 & &&& q_0\\  
	-1 & 2 &&& q_1\\
   & -1 \\ 
	&    & \ddots & & \vdots \\
	&    &       &  2 & q_{k-2} \\
	&    &       &  -1 & q_{k-1}
\end{bmatrix}\]
convient.

On peut alors par exemple utiliser l'algorithme nearest plane de Baibai 
à partir de cette base pour déchiffrer. Notons que des bornes sur 
les vecteurs de la décomposition de Gram-Schmidt de cette matrice sont données
dans~\cite{EC:MicPei12}, ce qui peut-être intéressant car cela est lié 
au domaine fondamental utilisé par l'algorithme.

\end{subsubsection}

	
\end{subsection}
	\begin{subsection}{Opérations homomorphes}
	On rappelle que $\vec{v}$ est de la forme $\textbf{PowersOf2}(\vec{s})$ et que donc $\textbf{Flatten}(A) \cdot \vec{v} = A \times \vec{v}$ pour tout $A$.
	
	\paragraph{}
	\textbf{MultConst}
	\flushleft
	
	\textbf{Entrée} : Cet algorithme prend en entrée les paramètres du système, un chiffré $C \in \ZZq^{N \times N}$ d'un message $\mu$ et une constante $\alpha \in \ZZq$. \\
	\textbf{Sortie} : Cet algorithme retourne un chiffré de $\alpha \cdot \mu$.\\
	\textbf{Algorithme} : On calcule $M_{\alpha} = \textbf{Flatten}(\alpha \times I_N)$ puis l'on renvoie $\textbf{Flatten}(M_{\alpha} \times C)$.
	\begin{proof}
	\begin{align*}
	\textbf{MultConst}(C, \alpha) \times \vec{v} &= M_{\alpha} \times C \times \vec{v} \\
	&= M_{\alpha} \cdot (\mu * \vec{v} + \vec{e}) \\
	&= M_{\alpha} \times \mu * \vec{v} + M_{\alpha} \times \vec{e} \\
	&= \alpha * \mu * \vec{v} + M_{\alpha} \times \vec{e}
	\end{align*}
	\end{proof}
	\textbf{Erreur} : Le chiffré à une erreur $e_2 = M_\alpha \times
		\vec{e}$. 
		\[\norm{e_2} \leq N \norm{e_1}\]
	\paragraph{}
	\textbf{Add}
	\flushleft
	
	\textbf{Entrée} : Cet algorithme prend en entrée les paramètres du système et deux chiffrés $C_1, C_2 \in \ZZq^{N \times N}$ des messages $\mu_1, \mu_2 \in \ZZq$.\\
	\textbf{Sortie} : Cet algorithme retourne un chiffré de $\mu_1 + \mu_2$.\\
	\textbf{Algorithme} : On calcule et on retourne $\textbf{Flatten}(C_1 + C_2)$.
	\begin{proof}
	\begin{align*}
	\textbf{Add}(C_1, C_2) \times \vec{v} &= (C_1 + C_2) \times \vec{v} \\
	&= (\mu_1 * \vec{v} + \vec{e_1}) + (\mu_2 * \vec{v} + \vec{e_2}) \\
	&= (\mu_1 + \mu_2) * \vec{v} + \vec{e_1} + \vec{e_2}
	\end{align*}
	\end{proof}
\textbf{Erreur} : Le chiffré à une erreur $e_3 = \vec{e_1} + \vec{e_2}$.
\[\norm{e_3} \leqslant \norm{e_1} + \norm{e_2}\]
	\paragraph{}
	\textbf{Mult}
	\flushleft
	
	\textbf{Entrée} : Cet algorithme prend en entrée les paramètres du système et deux chiffrés $C_1, C_2 \in \ZZq^{N \times N}$ des messages $\mu_1, \mu_2 \in \ZZq$.\\
	\textbf{Sortie} : Cet algorithme retourne un chiffré de $\mu_1 * \mu_2$. \\
	\textbf{Algorithme} : On calcule et on retourne $\textbf{Flatten}(C_1 \times C_2)$.
	\begin{proof}
	\begin{align*}
	\textbf{Mult}(C_1, C_2) \times \vec{v} &= (C_1 \times C_2) \times \vec{v} \\
	&= C_1 \times (\mu_2 * \vec{v} + \vec{e_2}) \\
	&= \mu_2 * C_1 \times \vec{v} + C_1 \times \vec{e_2} \\
	&= \mu_2 * (\mu_1 * \vec{v} + \vec{e_1}) + C_1 \times \vec{e_2} \\
	&= (\mu_1 * \mu_2) * \vec{v} + \mu_2 * \vec{e_1} + C_1 \times \vec{e_2}
	\end{align*}
	\end{proof}
	\textbf{Erreur} : Le chiffré à une erreur $e_3 = \mu_2 * \vec{e_1} + C_1 \times \vec{e_2}$. La matrice $C_1$ étant de la forme $\textbf{Flatten}(c_1)$, elle ne contient que des 0 et des 1. 
\[\norm{e_3} \leqslant \mu_2 \norm{e_1} + N \norm{e_2} \]
	\paragraph{}
	\textbf{NAND}
	\flushleft
	
	\textbf{Entrée} : Cet algorithme prend en entrée les paramètres du système et deux chiffrés $C_1, C_2 \in \ZZq^{N \times N}$ des messages $\mu_1, \mu_2 \in \{ 0,1\}$.\\
	\textbf{Sortie} : Cet algorithme retourne un chiffré de $\overline{(\mu_1 \land \mu_2)} = 1 - \mu_1 * \mu_2$. \\
	\textbf{Algorithme} : On calcule et on retourne $\textbf{Flatten}(I_N - C_1 \cdot C_2)$.
	\begin{proof}
	\begin{align*}
	\textbf{NAND}(C_1, C_2) \times \vec{v} &= (I_N - C_1 \times C_2) \times \vec{v} \\
	&= \vec{v} - \textbf{Mult}(C_1, C_2) \\
	&= \vec{v} - (\mu_1 * \mu_2) * \vec{v} - \mu_2 * \vec{e_1} + C_1 \times \vec{e_2} \\
	&= (1 - \mu_1 * \mu_2) * \vec{v} - \mu_2 * \vec{e_1} - C_1 \times \vec{e_2}
	\end{align*}
	\end{proof}
	\textbf{Erreur} : Le chiffré à une erreur 
	$e_3 = -(\mu_2 * \vec{e_1} + C_1 \times \vec{e_2})$. On est dans 
	un contexte similaire à $\textbf{Mult}(C_1, C_2)$, mais 
	$\mu_2$ est ici égal à 0 ou 1.
\[\norm{e_3} \leqslant \norm{e_1} + N \norm{e_2} \leqslant (N+1) \max(\norm{e_1}, \norm{e_2})\]
	\end{subsection}
\end{section}

\begin{section}{Analyse du cryptosystème: sécurité, profondeur des circuits}
	\begin{subsection}{Sécurité du cryptosystème}

	\begin{definition}{Distance statistique}
	Soit $X$ et $Y$ deux variables aléatoires supportée par
	un ensemble $\mathcal{V}$ et à valeur 
	dans un groupe abélien $G$. On définit la distance 
	statistique entre $X$ et $Y$, notée $\SD(X,Y)$, 
	comme étant la somme:
	\[ \frac{1}{2} \sum_{v \in \mathcal{V}} |\mathbb{P}(X = v) -
	\mathbb{P}(Y = v)| \]

	De plus, on dira que deux familles ${\{X_i\}}_{i \in \NN}$, ${\{Y_i\}}_{i \in \NN}$
	de distributions sont statistiquement indistinguables si 
	la fonction 
	\[ i \longrightarrow \SD(X_i, Y_i) \]
	est négligeable.

	% \begin{prop}
	% Pour $X$ et $Y$ définies comme précédemment, on a:
	% \[\SD(X,Y) = \max_{\mathcal{W}\subset\mathcal{V}} | \PP(X\in\mathcal{W}) - \PP(Y\in\mathcal{W})| \]
	% \end{prop}
	% \begin{proof}
	% \end{proof}
	\end{definition}
	\begin{prop} \label{sd_add}
	Soit ${(X_i)}_{1\leq i \leq n}$ et resp .${(Y_i)}_{1\leq i\leq n}$
	deux n-uplets de distributions indépendantes.
	\[ \SD\left((X_1,\cdots,X_n), (Y_1, \cdots, Y_n)\right) \leq \sum_{i=1}^n \SD(X_i,Y_i) \]
	\end{prop}
	\begin{proof}
	Montrons le pour $n = 2$, la suite se déduisant par récurrence.
	On a:
	\begin{align*}
		&\SD\left((X_1,X_2),(Y_1,Y_2)\right) \\
		&= \frac{1}{2} \sum_{(u,v)}\left| \PP(X_1=u)\PP(X_2=v) -
		\PP(Y_1=u)\PP(Y_2=v) \right| \\ 
		&\leq 
		\frac{1}{2} \sum_{(u,v)}\left| \PP(X_1=u)\left(\PP(X_2=v) - \PP(Y_2 = v)\right)
		- (\PP(X_1 = u) - \PP(Y_1=u))\PP(Y_2=v) \right| \\
		&\leq
		\frac{1}{2} \sum_{(u,v)} \PP(X_1 = u)
		\left|\left(\PP(X_2=v) - \PP(Y_2 = v)\right) \right| + 
		\frac{1}{2} \sum_{(u,v)}\PP(Y_2=v)\left|(\PP(X_1 = u) - \PP(Y_1=u)) \right| \\
		&= \SD(X_1, Y_1) + \SD(X_2, Y_2)
	\end{align*}
	\end{proof}
	\begin{prop}
	Si deux familles de distributions sont statistiquement
	indistinguables, elles sont calculatoirement indistinguables.
	\end{prop} 

	Le lemme suivant correspond au Claim 5.2 présent dans~\cite{STOC:Regev05}
	et nous sera utile pour la suite.

	\begin{lemme}
	Soit $G$ un groupe abélien fini. Pour $r > 1$ et 
	$\mathcal{F} \subset (g_1, \ldots, g_r) \in G^r$, on
	note
	 $s_\mathcal{F}$ la distribution aléatoire qui à un aléa
	 fait correspondre la somme $\sum_{i\in X} g_i$ pour un
	 sous-ensemble choisi de façon uniforme  $X\subset \llbracket
	 1, r \rrbracket$. 
	 D'autre part, on considère la distribution uniforme
	 $U$ sur $G$
	 Alors, on a: 
	 \[\mathbb{E}_{\mathcal{F}\subset G^r}(SD(s_\mathcal{F},U)) \leq 
	   \sqrt{\frac{|G|}{2^r}}\]
	 Notamment, 
	 \[\mathbb{P}\left(SD(s_\mathcal{F},U) \geq
		 \sqrt[\leftroot{-3}\uproot{8}4\:]{\frac{|G|}{2^r}} \right) \leq
		 \sqrt[\leftroot{-3}\uproot{8}4\:]{\frac{|G|}{2^r}}
	 \]
	\end{lemme}
	\begin{proof}
		Remarquons que:
	\begin{align*}
		\sum_{h\in G} {\PP(s_F = h)}^2 &= 
		\PP\left(\sum_i b_i g_i = \sum_i b'_i g_i\right) \\
		& \leq \frac{1}{2^l} + \PP
		\left( \sum_i b_i g_i = \sum_i b'_i g_i |\:\: (b_i)_i \neq
		(b'_i)_i \right)
	\end{align*}
	Or, pour $(b_i)_i \neq (b'_i)_i$, 
		\[ \PP\left((g_i)_i\: :\: \sum_i b_i g_i = \sum_i b'_i
		g_i\right) = \frac{1}{|G|}\]
	D'où on déduit que:
	\[\mathbb{E}_\mathcal{F}\left(\sum_h {\PP(s_{\mathcal{F}} =
	h)}^2\right) \leq \frac{1}{2^l} + \frac{1}{|G|} \]

	Ce qui implique que:

\begin{align*}
\mathbb{E}_\mathcal{F}\left[ \sum_h\Big| \PP(s_F = h) - 1/|G|\Big| \right] 
&\leq
\mathbb{E}_\mathcal{F}
\left[ 
	{|G|}^{1/2}
	{ \left( \sum_h {\left(\PP(s_F = h) - 1/|G|\right)}^2
	\right)}^{1/2}
\right] 
\\
&= \sqrt{|G|} \:\:
\mathbb{E}_\mathcal{F}
	\left[ 
	{\left( \sum_h {\PP(s_F = h)}^2 - 1/|G|\right)}^{1/2} 
	\right] \\
&\leq \sqrt{|G|} \:\:
{\left( 
	\mathbb{E}_\mathcal{F}\left[ \sum_h  {\PP(s_F = h)}^2  \right]
- \frac{1}{|G|}
\right)}^{1/2} \\
	&\leq \sqrt{\frac{|G|}{2^l}}
\end{align*}

\end{proof}

	\begin{cor}
	Soit $G$ un groupe abélien fini. Pour $r > 1$ et 
	$\mathcal{F} \subset (g_1, \ldots, g_r) \in G^r$, on
	note
	 $s_\mathcal{F}$ la distribution aléatoire qui à un aléa
	 fait correspondre la somme $\sum_{i\in X} g_i$ pour un
	 sous-ensemble choisi de façon uniforme  $X\subset \llbracket
	 1, r \rrbracket$. 
	 Considérons alors le n-uplet $S_\mathcal{F} = (X_1, \ldots, X_r)$ où 
	 les $X_i$ sont indépendants de même loi
	 $s_\mathcal{F}$.
	 D'autre part, on considère la distribution uniforme
	 $U$ sur $G^r$
	 Alors, on a: 
	 \[\mathbb{E}_{\mathcal{F}\subset G^r}(SD(s_\mathcal{F},U)) \leq 
	   \sqrt{r^2\frac{|G|}{2^r}}\]
	 Notamment, 
	 \[\mathbb{P}\left(SD(s_\mathcal{F},U) \geq
		 \sqrt[\leftroot{-3}\uproot{8}4\:]{r^2\frac{|G|}{2^r}} \right) \leq
		 \sqrt[\leftroot{-3}\uproot{8}4\:]{r^2\frac{|G|}{2^r}}
	 \]
	\end{cor}
	\begin{proof}
		Découle directement de la proposition précédente ainsi que de
		la proposition \ref{sd_add}
	\end{proof}

	\begin{prop}
	Supposons avoir pris des paramètres $(n, q, \chi, m)$
	tels que l'hypothèse $\LWE$ soit vraie. Alors pour $\tau >0$
	et $m > (1+\tau)(n+1)\log(q)$ et $m = \mathcal{O}(n\log(q))$ , la distribution jointe
	$(A, RA)$ est calculatoirement indistinguable de la
	distribution uniforme sur $\ZZq^{m \times (n+1)} \times \ZZq^{N
	\times (n+1)}$
	\end{prop}
	\begin{proof}
	On peut deja voir que comme $A$ est calculatoirement
	indistinguable de $U$, $(A, RA)$ l'est de $(U,RU)$ car on 
	peut facilement créer $(A, RA)$ (resp. $(U, RU)$) à partir de
	$A$ (resp. $U$).

	Il nous faut donc monter que $\mathcal{D}_1 = (U, RU)$ est calculatoirement
	indistinguable de $\mathcal{D}_2 = (U, V)$ où $V$ est uniforme.
	
	On peut alors utiliser le lemme précédent avec $G = \ZZq^{n+1}$
	et $r = m$ afin de voir qu'il echiste une constante $\lambda > 0$
	telle que:
	\[\mathbb{E}_{\mathcal{U}\subset \ZZq^{m\times n+1}}(SD(RU,V)) \leq 
		\sqrt{m^2\frac{q^{n+1}}{2^m}}\leq \lambda
	n\log(q)\sqrt{\frac{1}{q^{\tau(n+1)}}}=: f(n) \]
	Et, notant $Y = \{ U: \SD(RU,V) \geq \sqrt{f(n)}\}$, on obtient:
	\[\PP(U \in Y) \leq \sqrt{f(n)} \]
	où $f$ est négligeable en $n$. 



	
	Soit $(x,y)\in \ZZq^{m \times (n+1)} \times \ZZq^{m \times (n+1)}$
	\begin{align*}
	&\left|\PP(D_1 = (x,y)) - \PP(D_2 = (x,y))\right| \\ &\leq \PP(x\in Y)\:
	\Big|\PP(D_1 
	= (x,y)| x\in Y) - \PP(D_2 = (x,y)|x\in Y)\Big| + \PP(x \not\in Y)  \\
	&\leq |\PP(D_1 = (x,y) | x \in Y) - \PP(D_2 = (x,y)|x \in Y)| + \sqrt{f(n)} \\
	&\leq 2\sqrt{f(n)} 
	\end{align*}
	
	Ainsi, il n'est pas possible qu'un automate $\mathcal{A}$
	polynomial probabiliste puisse distinguer 
	$\mathcal{D}_1$ de $\mathcal{D}_2$ car elle sont statistiquement 
	indistinguables.
	\end{proof}
	\begin{thm}
	\label{ind_cpa}
	Sous les hypothèses de la proposition précédente, le
	cryptosysteme est IND-CPA.
	\end{thm}
	\begin{proof}
	Comme un automate polynomial probabiliste ne peut pas distinguer
	$A_{s, \chi}$ de la distribution uniforme, on peut supposer que la
	clef publique $A$ est uniforme.

	Considérons alors un chiffré 
	\[C = \flatten\left(\mu \cdot \id_N + \bitdecomp(R\cdot A)\right) \in
	\ZZq^{N\times N}\]

	On a:
	\[ \bitdecomp^{-1}(C) = \mu * \bitdecomp(\id_N) + R\cdot A\]

	Par la proposition précédente, un automate polynomial probabiliste $\mathcal{A}$
	ne peut pas distinguer $R\cdot A$ d'une matrice uniforme. On peut donc
	supposer que $R\cdot A$ est uniforme, et que le chiffrement est donc
	un one-time pad.

	On en déduit qu'il n'existe pas d'automate polynomial probabiliste
	$\mathcal{A}$ permettant de déchiffrer efficacement les chiffrés de ce cryptosystème.
	\end{proof}
	
	\end{subsection}

	Nous allons maintenant nous intéresser au choix de paramètres de notre
	cryptosystème. C'est une étape cruciale de la mise en place de celui-ci,
	car ce sont eux qui détermineront les degrés de sécurités et la profondeur des circuits calculables.
	
	Sur ce point, deux approches sont possibles: une étude asymptotique ou
	bien une étude \og concrète \fg des paramètres.

	Nous nous interesserons ici uniquement à l'algorithme de déchiffrement
	\textbf{Dec}, nous ne considèrerons donc que des chiffrés de $0$ ou
	$1$ et nous interesserons principalement à la profondeur maximale 
	possible de NAND permettant encore de déchiffrer.

	\begin{subsection}{Choix asymptotique de paramètres pour un leveled GSW}
	\label{param_leveled}
	Les indications faites dans l'article original \cite{EPRINT:GenSahWat13} n'étaient pas très explicites; nous nous inspirons donc ici du travail
	effectué par Shai Haleva dans \cite{halevi} pour avoir un jeu de
	contraintes nous permettant des paramètres permettant de faire un
	leveled GSW. 

	Rappelons que nous avons déjà pris pour hypothèse que le problème DLWE est difficile avec les paramètres suivants:
	\[ q \approx 2^{n^\epsilon}\quad \alpha q = n\quad \text{$m$ polynomial en $n$}\]
	De plus, le théorème~\ref{ind_cpa} necessite d'avoir 
	\[m > (1+\tau)(n+1)\log(q) \]
	pour un $\tau > 0$ pour que le cryptosystème soit IND-CPA.  Nous prenons alors: 
	\[ m = 2(n+1)(\lfloor \log(q) \rfloor + 1) = 2(N + \bnorm{q}) < 2
	\bnorm{q} (N+1)\]
	Enfin, nous devons aussi respecter la condition de longueur pour pouvoir appliquer une profondeur de $L$ NAND à notre chiffré:
	\[q > 8nm (1 + N)^L \]

En utilisant notre inégalité sur $m$  dans cette dernière équation, on voit alors qu'il nous suffit d'avoir:

\begin{equation}
q > 16 {(1+N)}^{L+2}
\end{equation}
car alors:
\begin{equation}
q> 8 (8 N (N+1)) {(1 + N)}^{L} > 8 nm {(1 + N)}^L \\
\end{equation}

Ceci nous ammène à premier jeu de contraintes: 
\[ \begin{cases}\alpha  = n \cdot 2^{-n^\epsilon}=  \\
	q = \lceil 2^{n^\epsilon}\rceil\\ 
	m = 2(N + \bnorm{q}) \\  
	n^\epsilon > 4 + (L+2) \log\left( 1 + N\right)
	\end{cases} \]

Nous allons encore simplifier la dernière contrainte en utilisant:
\begin{align*} (L+2) \log\left( 1 + N\right) &\leq (L+2) (2 + \log(N)) \\
&\leq (L+2) (2 + \log(n) +  \log(\bnorm{q})) \\
&\leq (L+2) (3 + \log(n) + \log(\log(q))
&\leq 2 L \log(n) \quad \text{pour $n$ assez grand}
\end{align*}

Ce qui au final, nous donne:
\[ \begin{cases}
	\alpha  = n \cdot 2^{-n^\epsilon}=  \\
	q = \lceil 2^{n^\epsilon} \rceil\\ 
	m = 2(N + \bnorm{q}) \\  
	n^\epsilon > 2 L \log(n)
	\end{cases}  \]
Il nous reste donc à trouver une valeur pour $n$, dépendant du paramètre de sécurité $\lambda$.  En posant $n = \rho^{1/\epsilon}$, on voit que la dernière contrainte devient:
\[ \frac{\rho}{\log(\rho)} > \frac{2 L}{\epsilon}  \]
Et en prenant $\rho = \text{cst} a \log(a)$, cela devient:
\[\log(a)> \frac{2 L}{\epsilon\:\text{cst}\: a}\:(\log(a) + \log(\log(a)) + \log(cst))   \]
qui est vérifiée pour $a = L$ et $cst = 4/\epsilon$ car alors, on obtient:
\begin{align*}
& \log(L)> \frac{1}{\text{2}}\:(\log(L) + \log(\log(L)) + 2 - \log(\epsilon)) \\
&\Leftrightarrow  \log(L) - \log(\log(L)) - 2 > \log(\frac{1}{\epsilon})
\end{align*}
ce qui est vrai pour $L$ assez grand.  

\begin{align*}
\Leftrightarrow  \log(L) - \log(\log(L)) - 2 > 0
\end{align*}
qui est une fonction croissante en $L$, et positive dés $L = 20$.  Notons enfin que ce choix de valeur $n$ est indépendant du paramètre de sécurité $\lambda$, on peut donc prendre le maximum entre les deux
valeurs pour plus de cohérence.  

On en déduit le théorème suivant:
\begin{thm}{leveled GSW}
Pour $L$ et $\lambda$ assez grands, et sous l'hypothèse sur DWLE~(page \pageref{hyp_dwle}), les paramètres suivants permettent de faire une profondeur de NAND de $L$
et rendent le cryptosystème GSW IND-CPA:
\[ \begin{cases} 
	n = \max\left(\lambda, \lceil 4/\epsilon \log(L) \log(\log(L))
	\rceil\right)  \\
	\alpha  = n \cdot 2^{-n^\epsilon}  \\
	q = \lceil 2^{n^\epsilon} \rceil\\ 
	m = 2(N + \bnorm{q}) \\  
	\end{cases}  \]
\end{thm}
Remarquons toutefois qu'il s'agit uniquement de paramètres \og théoriques \fg
étant donné que la sécurité n'est assurée qu'asymptotiquement. De plus, leur
taille est très grande, comme le montre TRUC




	\end{subsection}

	\begin{subsection}{Choix de paramètres concrets pour un leveled GSW}
	\begin{subsubsection}{Présentation de lwe\_estimator}
	
	Initialement utilisé dans l'article
	\cite{EPRINT:AlbPlaSco15},\path{lwe_estimator}
	(disponible à l'adresse \cite{estimator}) est un module de sagemath actuellement maintenu par
Martin Albrecht et 
	destiné à estimer la résistance face à diverses attaques de paramètres précis pour
	le problème de learning with error.

	Nous avons pensé qu'il pouvait être interessant de l'utiliser afin de
	voir si nous pouvions trouver des paramètres offrants une sécurité
	concrète, et non uniquement une famille de paramètres offrant une
	sécurité asymptotique.
	
	
	\paragraph{}
	\textbf{estimate\_lwe :}

	\paragraph{}
	Pour estimer la résistance de paramètres choisis sur un panel
	d'attaques, on utilise la fonction \path{estimate_lwe} dont une sortie
	typique est:
	
	\flushleft
	
	\begin{lstlisting}
estimate_lwe(n, alpha=None, q=None, secret_distribution=True, m=oo,
             reduction_cost_model=reduction_default_cost,
             skip=("mitm", "arora-gb", "bkw"))
        \end{lstlisting}
	
	\flushleft
	
	Cette dernière prends en arguments les paramètres usuels de LWE, n, $\alpha$ et q, ainsi que
	d'autres arguments optionnels et rend plusieurs résultats dont le sens n'est pas forcément évident.
	En fait, elle retourne tout un ensemble de variables pour chaque attaque vérifiée. Le module
	contient 6 attaques différentes, mais n'en testera que trois par defaut. Cela peut être modifié
	lorsque l'on appelle la fonction \path{estimate_lwe} via l'argument skip.
	
	\flushleft
	
	\begin{figure}
	\label{fig:seal_estimate}
	\begin{lstlisting}[mathescape=true]
sage: load("estimator.py")
sage: n = 2048; q = 2^60 - 2^14 + 1; $\alpha$ = 8/q; m = 2*n
sage: _ = estimate_lwe(n, $\alpha$, q, secret_distribution=(-1,1), 
	  reduction_cost_model=BKZ.sieve, m=m)
usvp: rop: =2^115.5,  red: =2^115.5,  $\delta_0$: 1.004975,  
      $\beta$:  288,  d: 4013,  m: 1964
 dec: rop: =2^127.1,  m:  =2^11.1,  red: =2^127.1,  $\delta_0$: 1.004663,  
      $\beta$: 318, d: 4237,  
      babai: =2^114.8,  babai_op: =2^129.9,  repeat: 7,  $\epsilon$: 0.500000
dual: rop: =2^118.4,  m:  =2^11.0,  red: =2^118.4,  $\delta_0$: 1.004864,  
      $\beta$: 298,  
      repeat:  =2^58.8,  d: 4090,  c:    3.909,  k: 30, postprocess: 13
	\end{lstlisting}
	\caption{Analyse de sécurité des paramètres tirés de la librairie SEAL}
	\end{figure}
	FIGURE 1 DEVRAIT ÊTRE ICI \\
	\flushleft
	
	Les variables rendues pour chaque attaques ne sont pas toutes utiles, certaines étant strictement internes à ces attaques. Les trois variables qui nous intéresse sont : "rop", "m" et "mem" :
	
	\begin{itemize}
	\item "rop" (ring operations) est une estimation du nombre d'opérations à effectuer afin de résoudre ce cas de LWE avec cette attaque.
	
	\item "mem" (memory) est une estimation de la mémoire qui sera exploitée.
	
	\item "m" indique le nombre d'échantillons nécessaires pour résoudre le problème avec ces valeurs de
	"rop" et "mem". A noter que l'on peut limiter le nombre d'échantillons disponibles pour l'attaquant
	lorsqu'on appelle \path{estimate_lwe}.
	\end{itemize}

	le module \path{sage.crypto.lwe} contient des choix de paramètres pour
	le problème LWE, dont deux qui utilisent la gaussienne discrète
	\footnote{un autre utilise la distribution uniforme}
	et que nous allons présenter ici. Notons que nous avons pour cela 
	regardé les codes 
	sources des fonctions sage, disponibles dans leur github (\cite{sage}).
	\end{subsubsection}

	\begin{subsubsection}{Proposition de choix sécurité pour très faible profondeur}
	\paragraph{}
	En utilisant les paramètres suivants, tirés de l'API SEAL:
	\[n = 2048\quad q = 2^{60} - 2^{14} + 1 \quad \alpha = \frac{8}{q}\quad m = 2n \]
	On voit que l'estimation proposée par lwe indique  l'attaque la plus rapide demande $2^{115}$ opérations de base dans l'anneau $\ZZq$, soit un facteur de sécurité de 115. De plus, l'équation~\eqref{depth} indique qu'une profondeur de NAND $L=3$ est possible. \\
	On a ici un secret de 15 Ko, une clé publique de 7.6 Mo et des chiffrés de 13 Go.

	\paragraph{}
	En utilisant les paramètres suivants, tirés de \cite{cryptoeprint:2015:755}:
	\[n = 804\quad  q = 2^{31} - 19\quad \alpha = \frac{\sqrt{2\pi}*57}{q} \quad m = 4972\]
	On voit que l'estimation proposée par lwe indique  l'attaque la plus rapide demande $2^{129}$ opérations de base dans l'anneau $\ZZq$, soit un facteur de sécurité de 129. De plus, l'équation~\eqref{depth} indique qu'une profondeur de NAND $L=1$ est possible. \\
	On a ici un secret de 3 Ko, une clé publique de 5 Mo et des chiffrés de 2 Go.
	\end{subsubsection}

	\end{subsection}
\end{section}

\pgfmathsetmacro{\ha}{0}
\pgfmathsetmacro{\va}{0}
\node  (before) at ($(\ha, \va)$) {\small$C = D + \text{\textcolor{Orange}{erreur}}$}; 
\node  (after) at ($(\ha+8, \va)$) {\small$C_{\text{new}} =D + \text{\textcolor{Green}{erreur}}$}; 
\node  (clear) at ($(\ha+4, \va)$) {$\mu$}; 
\node  (e) at ($(\ha+2.5, \va+0.5)$) {$\text{Decrypt}(\vec{sk}, C)$}; 
\node  (d) at ($(\ha+5.5, \va+0.5)$) {$\text{Encrypt}(pk, \mu)$}; 
\draw[->] (before) to (clear);
\draw[->] (clear) to (after);

\begin{section}{Implémentation d'un FHE avec bootstrapping \og{} jouet\fg{}}
\begin{subsection}{Présentation de notre arborescence}
Nous avons concu une implémentation simple 
du cryptosystème GSW en sagemath, située dans le dossier
\path{GSW_implementation}. Celui-ci contient quatres dossiers:
\begin{itemize}
\item \path{GSW_scheme} contenant l'implémentation de GWS;
\item \path{analysis} contenant des fonctions permettant 
	de tester les fonctionnalités de notre implémentation 
	ou encore de voir les performances 
	en terme de sécurités de certains choix de paramètres;
\item \path{unitary_test} contenant des tests assurant le bon fonctionnement
	des fonctions codées dans les autres dossiers; 
\item \path{lwe_estimator} contient les fichiers sources de l'API
	lwe\_estimator que nous avons présenté dans ce rapport; 
\end{itemize}

Nous proposons ici de faire une revue rapide des trois premiers dossiers. Notez que pour
\og attacher \fg avec sage un des sources, il faut rester à la racine pour
éviter des problèmes liés à l'utilisation de chemin relatifs pour les imports.


\begin{subsubsection}{GWS\_scheme}
Ce dossier contient les fichiers suivants:
\begin{itemize}
\item \path{GSW_scheme.sage} contient les fonctions principales de GWS
dont \path{setup}, \path{encrypt} et les 3 algorithmes de
déchiffrements que sont \path{basic_decrypt}, \path{mp_decrypt}
et \path{mp_all_q_decrypt}. Il contient aussi différentes variables
globales, dont \path{decrypt} permettant d'indiquer quel est
l'algorithme de dechiffrement par défault et les variables 
\path{bs_foo} indiquant quels paramètres sont utilisés lorsque 
on utilise des fonctins avec bootstrapping;
\item \path{auxilliary_functions.sage} contient l'implémentation 
	des diverses fonctions auxilliaires utilisées pour chiffrer 
	et déchiffrer les messages, comme par exemple \path{flatten};
	ou encore une implémentation du nearest plane de Babai;
\item \path{params_maker.sage} contient diverses fonctions permettant, à 
	partir d'un $n$, de retourner des paramètres $n, q, \chi, m$
	utilisés par le cryptosystème. Le fichier \path{GWS_scheme.sage}
	contient une variable globale \path{params_maker}
	permettant de fichier celui qu'utilise la fonction
	\path{sectup};
\item \path{homomorphic_functions.sage} contient la version homomorphe
	d'opérations de base comme la somme, ou encore le NAND;
\item \path{bootstrapping.sage} contient les fonctions necessaires pour
	effectuer l'algorithme \path{basic_decrypt} homomorphiquement
	(il s'agit de la fonction \path{h_basic_decrypt}). On y trouve
	donc notamment diverses façon de sommer homomorphiquement
	des listes de chiffrés de $0$ et de $1$. Notez que la fonction
	\path{bootstrapping_arguments} permet de faire un bootstrapping
	en retournant une valeur \og mise à jour \fg des chiffrés
	passés en argument, cette fonction est notamment utilisée dans 
	\path{analysis/h_circuits_with_bootstrapping.sage}.
\end{itemize}
\end{subsubsection} % GSW_scheme

\begin{subsubsection}{analysis}
Ce dossier contient les fichiers suivants:
\begin{itemize}
\item \path{depth_security.sage} contenant l'implémentation de GWS;
\item \path{circuits.sage} contenant des fonctions permettant 
	de tester les fonctionnalités de notre implémentation 
	ou encore de voir les performances 
	en terme de sécurités de certains choix de paramètres;
\item \path{clear_functions.sage} contient des versions \og en clair \fg
	de fonctions homomorphes, utilisées dans les circuits;
\item \path{h_circuits_with_bootstrapping.sage} contient des exemples de
	fonctions utilisant des bootstrappings. Elles permettent de voir 
	si, pour certaines fonctions $f$, appliquer $f$ homomorphiquement 
	sur des chiffrés revient au même que d'abord l'appliquer sur les 
	clairs puis chiffrer.
	On peut toutes les lancer en  
	utilisant  la fonction \path{all_circuit_without_bs};
\item \path{h_circuits_without_bootstrapping.sage} contient des examples de
	fonctions n'utilisant pas de bootstrappings. 
	fonctions utilisant des bootstrappings. Elles permettent de voir 
	si, pour certaines fonctions $f$, appliquer $f$ homomorphiquement 
	sur des chiffrés revient au même que d'abord l'appliquer sur les 
	clairs puis chiffrer.
	On peut toutes les lancer en  
	utilisant  la fonction \path{all_circuit_with_bs};
\item \path{circuits.sage} Contient des fonctions permettant d'écrire 
	sous forme de string des fonctions algébriques simples, ce qui est utilisé
	dans  \path{h_circuits_without_bootstrapping.sage}. Par exemple, on 
	peut écrire \path{abc|*c+a~bc} pour signifier la fonction
	\[(a,b,c) \mapsto c * (a + (b \text{NAND} c)) \];
\item \path{all_circuit_analysis.sage} contient la fonction
	\path{analysis_main} qui lance les différents circuits avec et sans
	bootstrappings des fichiers précédents. 
\end{itemize}
\end{subsubsection} % analysis

\begin{subsubsection}{unitary\_test}
Ce dossier contient un fichier \path{framework_test.sage} permettant de mettre 
en forme les sorties des différentes fonctions de test, puis un fichier 
de test correspondant à chaque fichier du dossier \path{GSW_scheme}.
Chacun de ses fichiers contient une fonction \path{test_main_FOO}
ne demandant aucun argument et permettant de lancer les différents 
tests qu'il contient. De plus, le fichier \path{all_main_test.sage}
contient une fonction \path{test_main} permettant de lancer toutes 
les fonctions de forme \path{test_main_FOO}. On peut donc se faire une
idée du travail réalisé sur les tests en la lancant.
\begin{itemize}
\item \path{depth_security.sage} contenant l'implémentation de GWS;
\item \path{circuits.sage} contenant des fonctions permettant 
	de tester les fonctionnalités de notre implémentation 
	ou encore de voir les performances 
	en terme de sécurités de certains choix de paramètres;
\item \path{clear_functions.sage} contenant des tests assurant le bon fonctionnement
	des fonctions codées dans les autres dossiers; 
\item \path{h_circuits_with_bootstrapping.sage} contient les fichiers sources de l'API
	lwe\_estimator que nous avons présenté dans ce rapport; 
\item \path{h_circuits_without_bootstrapping.sage} contient les fichiers sources de l'API
	lwe\_estimator que nous avons présenté dans ce rapport; 
\end{itemize}
\end{subsubsection} % analysis

\end{subsection}
\end{section}

\begin{section}{Des librairies pour du FHE}
Plusieurs librairies open-sources implémentant divers FHE sont disponibles. 
On peut notamment en trouver une liste sur HomomorphicEncryption.org \cite{homencrypt.org}, 
qui se décrit comme \og an open consortium of industry, government and academia to 
standardize homomorphic encryption \fg.

Nous proposons ici d'en évoquer deux:
\begin{itemize}
\item The Simple Encrypted Arithmetic Library (SEAL) \cite{seal}, dont nous avons tiré des paramètres  
\og réalistes \fg\footnote{Pas forcément pour nos machines et avec notre implémentation}
sécurisés et autorisant une profondeur de NAND non null (même si irréaliste: seulement 3);
\item The Gate Bootstrapping API \cite{TFHE} qui implémente une variation du cryptosystème GSW;
\end{itemize}

\begin{subsection}{The Simple Encrypted Arithmetic Library (SEAL)}
Acronyme de Simple Encrypted Arithmetic Library, SEAL (voir \cite{seal}) 
est une librairie écrite par le \og cryptography research group \fg de Microsoft, en C++ sous 
licence MIT. Elle se propose d'implémenter deux FHE de seconde génération: 
BVS \cite{EPRINT:FanVer12} et CKKS \cite{AC:CKKS17}.

Son installation est facile\footnote{Sur Linux debian 4.9.0-8-amd64, nous avons dû
utilister les backports debians pour avoir une version de cmake suffisament récente}
et il est directement possible de compiler un executable permettant de tester 
diverses fonctionnalitées de la librarie. De plus, la documentation
\cite{seal_manual_231},
malheureusement non à jour, indique quelques points théoriques autant du point 
de vue mathématique que des choix de représentation des données.
\end{subsection}

\begin{subsection}{The Gate Bootstrapping API}
Notre présentation s'appuie sur celle donnée dans la page officielle (voir \cite{TFHE})
 qui est claire et bien documentée.

l'API Gate Bootstrapping est une librairie open source utilisable en C, C++ et 
s'appuyant notamment sur des travaux de I. Chillotti, N. Gama, M. Georgieve et M. Izabachène 
(voir \cite{cryptoeprint:2017:430} et  \cite{cryptoeprint:2016:870}). 

Elle utilise une version modifiée du cryptosysteme GSW (\cite{C:GenSahWat13})
étudié dans notre rapport, et permettant aussi bien du LHE que du FHE. C'est
pourquoi nous allons parler plus en détail de celle-ci.

Ses performances sont interessantes; il est notamment indiqué dans la sous-section
4.2 de \cite{cryptoeprint:2016:870} que pour un ordinateur 64-bit simple coeur 
(i7-4930MX) cadencé à 3.00GHz, le bootstrapping se fait en un temps moyen de 52ms.
de clée de bootstrapping d'environ 24MO.

Pour arriver à de tels résultats, de nombreuses modifications et optimisations dans le codes 
ont été faites. Notamment, le problème sur lequel s'appuie le cryptosystème n'est plus 
LWE mais TFHE, présenté dans les librairies suscitées.

\begin{subsubsection}{Le problème TFHE}
\end{subsubsection} % Le problème TFHE
\begin{subsubsection}{Un exemple simple fourni par leur tutorial}
\end{subsubsection} % un exemple simple fourni par leur tutorial
\end{subsection} % TFHE
\end{section}

\begin{section}{Conclusion}
\end{section}


\newpage
\bibliographystyle{plain}
% abbrev0 give the more detailled references
\bibliography{cryptobib/abbrev0,cryptobib/crypto,aux_biblio}
\end{document}
