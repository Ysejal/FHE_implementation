\begin{section}{Présentation du cryptosystème}
	\begin{subsection}{L'idée générale}
		% Explication sans l'erreur, pourquoi c'est homomophique
	\end{subsection}
	\begin{subsection}{LWE : Un bref survol}
	Voici la définition du LWE, telle que donnée dans
	\cite{C:GenSahWat13} 

	\begin{definition}{Learning with Errors (LWE), version décisionnelle}
	Pour un paramètre de sécurité $\lambda$, soit $n = n(\lambda),q = q(\lambda)$ 
	des entiers et $\chi = \chi(\lambda)$ une
	distribution sur $\ZZ$.
	Le problème $\LWE$ concisite à devoir distinguer 
	deux distributions sur $\ZZq^{n+1}$ à partir 
	de $m = m(\lambda)$ valeurs qu'une des deux à produite.
	La première distribution crée des vecteurs $(\vec{a}_i,b_i) \in
	\ZZq^{n+1}$ uniforme.
	La deuxième utilise un $\vec{s} \in \ZZq^n$ tiré uniformément et 
	prend pour valeurs des vecteurs $(\vec{a}_i, b_i)$ pour lesquels:
		\[ b_i = \langle \vec{a}_i, \vec{s} \rangle + e_i \]
	où $e_i$ est crée obtenue par $\chi$.
	\end{definition}


	On peut aussi définir ce problème sous une forme matricielle, les 
	deux définitions étant équivalentes.

	\begin{definition}{Learning with Errors (LWE), version decisionnelle matricielle}
	En prenant les paramètres de la précédente définition
	Le problème $\LWE$ concisite à décider 
	si une matrice $A \in \ZZq^{m \times (n+1)}$ 
	est uniforme ou bien si il existe un vecteur $\vec{v} = (1
	-\vec{s})$ tel que $A \cdot \vec{v} \in \ZZq{m}$ est 
	crée à partir de $\chi^m$. Autrement dit, si les lignes 
	de $A$ sont de la forme $(\langle a, \vec{s} \rangle + e \vec{a})$ 
	pour $e$ créé avec $\chi$.
	\end{definition}

	Comme l'indique le théorème 1 de \cite{C:GenSahWat13}, il est
	possible de réduire le problème LWE à des problèmes sur des réseaux.

	Voyons ici informellement comment réduire LWE à un problème de type 
	SVP\footnote{short vector problem}. 

	Nous allons ici considérer le problème LWE calculatoire, dans lequel il
	faut trouver le vecteur $\vec{s}$ tel que:

	\[ A\cdot s = \vec{e} \mod q \]
	où les coordonnées de $\vec{e}$ sont créées par $\chi$.

	De façon équivalente, il faut trouver un vecteur $(*\quad\vec{s})$ tel
	que:
	\[ \begin{bmatrix}q & A \\ 0 &1 \end{bmatrix}\cdot
	   \begin{bmatrix}* \\ \vec{s} \end{bmatrix} =
	   \begin{bmatrix} \vec{e} \\ \vec{s} \end{bmatrix} \]
	Si la distribution $\chi$ créé de petites valeurs, on voit qu'on à
	alors trouvé un \og petit \fg vecteur du réseau engendré par les colonnes de 
	\[ \begin{bmatrix}q & A \\ 0 &1 \end{bmatrix} \]
	\end{subsection}
	\begin{subsection}{Définition du cryptosystème}
		% on le définit, sans trop en dire sur les paramètres 
	\end{subsection}
\end{section}
