\begin{section}{Présentation du cryptosystème}
	\begin{subsection}{L'idée générale}
	L'idée de ce cryptosystème consiste à prendre pour secret un certain vecteur $\vec{v} \in \ZZq^N$ pour certains paramètres $q, N \in \NN$, puis à chiffrer un message $m \in \ZZq$ à l'aide d'une matrice $C \in \ZZq^{N \times N}$ ayant $m$ pour valeur propre associée au vecteur propre $\vec{v}$. Autrement dit, avec :
\[C \cdot \vec{v} = m \vec{v}\: \mod q \]

	De là, il est facile de voir que pour $\lambda \in \ZZ$ et  $C_1$ et $C_2$ chiffrés de $m_1$ et $m_2$, on a :
	\begin{align*}
	& (C_1 + C_2) \cdot \vec{v} = (m_1 + m_2) \vec{v} \\
	& (C_1 \times C_2) \cdot \vec{v} = (m_1 + m_2) \vec{v} \\
	& (\lambda  C_2) \cdot \vec{v} = (\lambda m_1) \vec{v} 
	\end{align*}

	Toutefois, un tel système n'est pas sécurisé car $C$ n'a qu'un nombre fini de valeurs propres, et il semble donc facile de retrouver le secret $\vec{v}$.

	La solution consiste alors à ajouter du bruit au chiffré, c'est à dire à chiffrer $m\in \ZZq$ par une matrice $C \in \ZZ^{N \times N}$ telle que :
	\[ C \vec{v} = m \vec{v} + \vec{e} \]
	pour une \og petite \fg \ erreur $\vec{e}$. Si le vecteur $\vec{v}$ contient un grand coefficient $v_i$, on voit alors qu'il reste possible de retrouver $m$ avec
	\begin{align*}
	\frac{{(C \vec{v})}_i}{v_i} = \frac{m + e_i}{v_i}
	\end{align*}
	
	Nous verrons que pour de bons choix de paramètres, déchiffrer un tel message permet de résoudre une instance de LWE.

	Toutefois, l'ajout d'une erreur comporte ses inconvénients. Si nous revenons aux équations précédentes, en introduisant les erreurs $\vec{e}_i$ pour chiffrer  $m_i$ ($i\in \{1,2\}$), on obtient :

	\begin{align*}
	& (C_1 + C_2) \cdot \vec{v} = (m_1 + m_2) \vec{v} + (\vec{e}_1 + \vec{e}_2)\\
	& (C_1 \times C_2) \cdot \vec{v} = (m_1 * m_2) \vec{v} + C_1 \vec{e}_2 + m_2\vec{e}_1 \\
	& (\lambda  C_2) \cdot \vec{v} = (\lambda m_1) + \lambda e_i\vec{v} 
	\end{align*}

	Notamment, on voit que le terme $C_1 * \vec{e}_2$ peut être très grand même pour un petit $\vec{e}_2$. Nous verrons par la suite comment choisir nos paramètres, et notamment $\vec{v}$, afin de toujours pouvoir se ramener à des chiffrés $C \in \{0,1\}^{N \times N}$. De cette façon, on aura :
	\[ ||C_1 \vec{e}_2|| \leqslant ||\vec{e_2}|| \]
	




	\end{subsection}
	\begin{subsection}{Fonctions nécessaires au cryptosystème}
	Nous allons ici définir le cryptosystème GSW, en utilisant l'exposition originale faite dans \cite{EPRINT:GenSahWat13}, légèrement modifiée pour prendre en compte des remarques interessantes faites par Shai Halevi dans \cite{halevi}.

	A METTRE DANS LE 4.3 ? WTF ?
	\paragraph{}
	\textbf{BitDecomp}
	\flushleft

	\textbf{Entrée} : Cet algorithme prends en entrée un vecteur $\vec{a} = (a_1, ..., a_k) \in \ZZq^{k}$.

	\textbf{Sortie} : Cet algorithme retourne la décomposition binaire des éléments de $\vec{a}$ sous la forme d'un vecteur.

	\textbf{Algorithme} : Pour chaque $a_i$, on détermine sa représentation binaire avec les bits de faibles puissance à gauche et non à droite. On retourne la concaténation de ces représentations binaires sous la forme d'un vecteur.
	
	\paragraph{}
	\textbf{BitDecomp}$^{-1}$
	\flushleft

	\textbf{Entrée} : Cet algorithme prends en entrée un vecteur $\vec{a} = (a_{1,0}, ..., a_{1,l-1}, a_{2,0}, ..., a_{k,l-1})$.

	\textbf{Sortie} : Cet algorithme renvoie ($\sum\limits_{i=0}^{l-1} 2^{i} a_{1,i}, ..., \sum\limits_{i=0}^{l-1} 2^{i} a_{k,i})$.

	\textbf{Remarque} : Si tous les $a_{i,j}$ sont dans $\{ 0,1 \} $, cet algorithme inverse bien \textbf{BitDecomp}, cependant, sa définition ne le limite pas aux vecteurs $\in \{ 0,1\} ^{k\times l}$.

	\paragraph{}
	\textbf{Flatten}
	\flushleft

	\textbf{Entrée} : Cet algorithme prends en entrée un vecteur $\vec{a} = (a_{1,0}, ..., a_{1,l-1}, a_{2,0}, ..., a_{k,l-1})$.

	\textbf{Sortie} : Cet algorithme retourne un vecteur $\vec{b} = (b_{1,0}, ..., b_{1,l-1}, b_{2,0}, ..., b_{k,l-1})$ dont les éléments sont tous dans $\{ 0,1\} $.

	\textbf{Algorithme} : On calcule \textbf{BitDecomp}$^{-1}(\vec{a})$ et on obtient un vecteur $\vec{z} \in \ZZq^{k}$. On applique ensuite \textbf{BitDecomp} à $\vec{z}$ et l'on renvoie le résultat obtenu.
	
	\paragraph{}
	\textbf{PowersOf2}
	\flushleft

	\textbf{Entrée} : Cet algorithme prends en entrée un vecteur $\vec{a} = (a_1, ..., a_k) \in \ZZq^{k}$.

	\textbf{Sortie} : Cet algorithme renvoie ($a_1, 2 a_1, 2^{2} a_1, ..., 2^{l-1} a_1, a_2, ..., 2^{l-1} a_k)$.
	
	\begin{prop}
	Soient $\vec{a}$ et $\vec{b}$ dans $\ZZq^{k}$.

	On a $\langle \textbf{BitDecomp}(\vec{a}), \textbf{PowersOf2}(\vec{b}) \rangle = \langle\vec{a},\vec{b} \rangle$.
	\end{prop}

	\begin{proof}
	\begin{align*}
	\langle \textbf{BitDecomp}(\vec{a}), \textbf{PowersOf2}(\vec{b}) \rangle &= \sum\limits_{i=1}^{k} \sum\limits_{j=0}^{l-1} a_{i,j} * (2^{j} * b_i) \\
	&= \sum\limits_{i=1}^{k} b_i * \sum\limits_{j=0}^{l-1} (a_{i,j} * 2^{j}) \\
	&= \sum\limits_{i=1}^{k} b_i * a_i \\
	&= \langle\vec{a},\vec{b} \rangle.
	\end{align*}
	\end{proof}
	
	\begin{prop}
	Soient $\vec{a}$ dans $\ZZq^{k \times l}$ et $\vec{b}$ dans $\ZZq^{k}$.

	On a $\langle \vec{a}, \textbf{PowersOf2}(\vec{b}) \rangle = \langle \textbf{BitDecomp}^{-1}(\vec{a}), \vec{b}\rangle = \langle \textbf{Flatten}(\vec{a}),\textbf{PowersOf2}(\vec{b}) \rangle$.
	\end{prop}

	\begin{proof}
	\begin{align*}
	\langle \vec{a}, \textbf{PowersOf2}(\vec{b}) \rangle &= \sum\limits_{i=1}^{k} \sum\limits_{j=0}^{l-1} a_{j+li} * (2^{j} * b_i) \\
	&= \sum\limits_{i=1}^{k} b_i * \sum\limits_{j=0}^{l-1} (a_{j+li} * 2^{j}) \\
	&= \langle \textbf{BitDecomp}^{-1}(\vec{a}), \vec{b}\rangle \\
	\end{align*}
	Soit $c = \textbf{BitDecomp}^{-1}(\vec{a})$.
	\begin{align*}
	\langle \textbf{Flatten}(\vec{a}),\textbf{PowersOf2}(\vec{b}) \rangle &= \langle \textbf{BitDecomp}(\vec{c}),\textbf{PowersOf2}(\vec{b}) \rangle \\
	&= \sum\limits_{i=1}^{k} \sum\limits_{j=0}^{l-1} c_{i,j} * (2^{j} * b_i) \\
	&= \sum\limits_{i=1}^{k} b_i * \sum\limits_{j=0}^{l-1} (c_{i,j} * 2^{j}) \\
	&= \sum\limits_{i=1}^{k} b_i * c_i \\
	&= \langle \textbf{BitDecomp}^{-1}(\vec{a}), \vec{b}\rangle \\
	&= \langle \vec{a}, \textbf{PowersOf2}(\vec{b}) \rangle
	\end{align*}
	\end{proof}
	
	\end{subsection}
	\begin{subsection}{Définition du cryptosystème}
	On rappelle que les paramètres du système défini ici sont : le paramètre de dimension $n$, le modulus $q$, un modèle de distribution de l'erreur $\chi$ ainsi que $m$, qui, tout comme $n$ influera la taille des matrices manipulées.

	On note $l = \lfloor$log $q\rfloor + 1$ et $N = (n + 1)$ $l$.
		
	\paragraph{}
	\textbf{Setup}
	\flushleft

	\textbf{Entrée} : Cet algorithme prends en entrée $1^\lambda$ et $1^L$ avec $\lambda$ paramètre de sécurité et L paramètre de profondeur.

	\textbf{Sortie} : Cet algorithme retourne les paramètres $n, q, \chi, m$ du système. Le paramètre $L$ indique que les paramètres créés doivent permettre d'évaluer un circuit de \textbf{NAND} de profondeur $L$ tout en pouvant déchiffrer correctement. Nous montrerons qu'il est alors nécessaire de satisfaire la condition dite \og de longueur \fg  
	\[q > 8nm (1 + N)^L \]

	\textbf{Algorithme} : Nous verrons cela dans la section des considérations sur les choix des paramètres.
	
	\paragraph{}
	\textbf{KeyGen}
	\flushleft
	
	\textbf{Entrée} : Cet algorithme n'a besoin en entrée que des paramètres donnés par \textbf{Setup}.

	\textbf{Sortie} : Cet algorithme renvoie la clé secrète $\vec{s} \in \ZZq^{n+1}$ ainsi que la clé publique $A \in \ZZq^{m \times n}$ vérifiant la contrainte $A \cdot \vec{s} \leqslant n$.

	\textbf{Algorithme} :

	\textbf{clé secrète} : On génère un vecteur $t \in \ZZq^n$ avec $\chi$ et on définit la clé secrète comme $\text{\textbf{PowersOf2}}(\vec{s})$ pour $\vec{s} = (1, -t_1, ..., -t_n)$.

	\textbf{clé publique} : On génère aléatoirement uniforme $B \in \ZZq^{n \times m}$ et un vecteur $\vec{e}$ de m éléments choisis suivant la distribution $\chi$. On définit $\vec{b} = B \times \vec{t} + \vec{e}$. La clé publique est la matrice constituée de l'indentation de $\vec{b}$ considéré comme un vecteur colonne et de $B$.

	Si la contrainte $\norm{A \cdot \vec{s}} \leqslant n$ n'est pas vérifiée, on recrée un jeu de clés.

	\textbf{Taille} : Comme on l'a dit, $\vec{s} \in \ZZq^{n+1}$. Par définition, $q$, et donc tous élément de $\ZZq$, s'écrit en $l$ bits. $\vec{s}$ fait donc une taille de $l * (n+1) = N$ bits.

	D'autre part, $A \in \ZZq^{m \times n}$ donc A s'écrit en $l * n * m = N * (m - 1)$ bits.
	
	\paragraph{}
	\textbf{Encrypt}
	\flushleft
	
	\textbf{Entrée} : Cet algorithme prend en entrée les paramètres du système, la clé publique et un message $\mu \in \ZZq$.

	\textbf{Sortie} : Cet algorithme retourne le chiffré $C \in \ZZq^{N \times N}$ de $\mu$.

	\textbf{Algorithme} : On génère uniformément une matrice $R \in \{ 0,1\} ^{N \times m}$. Le chiffré est : $C = \textbf{Flatten}(\mu \times I_N + \textbf{BitDecomp}(R \times A))$.

	\textbf{Taille} : $C \in \ZZq^{N \times N}$ s'écrit en $l * N^2$ bits.
	
	\paragraph{}
	\textbf{Dec}
	\flushleft
	
	\textbf{Entrée} : Cet algorithme prend en entrée les paramètres du système, la clé secrète et un chiffré d'un message $\mu \in \{ 0,1\} $.

	\textbf{Sortie} : Cet algorithme retourne le clair du chiffré si l'erreur de ce dernier n'est pas trop élevée.

	\textbf{Algorithme} : On rappelle que les $l$ premiers coefficients de $\vec{v}$ sont les puissances de 0 à $l-1$ de 2. Soit $i \leqslant l$ tel que le i+1ème coefficient de $\vec{v}$, égal à $2^{i}$, soit compris entre $q/4$ et $q/2$, $q/2$ compris. On note $C_i$ la ième ligne de $C$. On calcule ensuite $x_i = \langle C_i, \vec{v} \rangle$ et on renvoie $\lfloor x_i/v_i \rceil$.


\paragraph{}
\begin{definition}
On appellera erreur d'un chiffré $C$ d'un message $\mu$ le vecteur $\vec{e}$ tel que
\[ C\cdot \vec{v} = \mu\, \vec{v} + \vec{e}. \]
\end{definition}

\begin{prop}
\label{dec}
\textbf{Dec} décrypte avec succès les chiffrés dont l'erreur $\vec{e}$ satisfait $\|\vec{e}\|_\infty < q/8$.
\end{prop}
\begin{proof}
	Dans ce cas, on a $x_i = \mu * v_i + e$ avec $\lvert e \lvert \leqslant \|\vec{e}\|_\infty$. Comme $\lvert v_i \lvert > \frac{q}{4}$, on a $\lvert \frac{e}{v_i} \lvert < 1/2$. On a donc $\lfloor \frac{x_i}{v_i} \rceil = \mu$.
\end{proof}
	\end{subsection}
	
\begin{subsection}{Autres algorithmes de déchiffrement}
	L'algorithme de déchiffrement que nous avons présenté fonctionne sans contraintes sur $q$ mais ne déchiffre que des chiffrés de 0 et de 1.

	Nous proposons ici une analyse un peu plus fine du déchiffrement pour montrer comment faire pour des chiffrés de n'importe quel élément de $\ZZq$.

	Pour cela, remarquons qu'en partant d'un chiffré  $C$ de $m \in \ZZq$, on a :
	\[ C \cdot \vec{v} = m \vec{v} + \vec{e} \mod q \]
	pour une erreur $\vec{e}$.
	
	En considérant l'équation sur les $l$ premières coordonées, on obtient :
	\[\vec{a} = m \vec{p} + \vec{e} \mod q\quad \text{où}\quad \vec{p} = (1\:2\:\cdots\:2^{l-1}). \]
 
	Notant $L = \Lambda(\vec{p}^t)$, on constate qu'on peut retrouver $m\vec{p}$ en trouvant le vecteur de $L$ le plus proche de $\vec{a}$.

	De cette idée, on déduit 2 algorithmes de déchiffrements supplémentaires, dépendant de la façon dont on résout le problème du vecteur le plus proche :

\begin{itemize}
\item \textbf{mp\_decrypt}, qui suppose que $q$ est une puissance de 2;
\item \textbf{mp\_all\_q\_decrypt}, sans hypothèses sur $q$.
\end{itemize}

\begin{subsubsection}{mp\_decrypt: $q$ est une puissance de $2$}
	L'algorithme, présenté dans \cite{EPRINT:GenSahWat13}, utilise le fait que $q = 2^l$.

	En regardant la dernière coordonnée de :
\[\vec{a} = m \vec{p} + \vec{e} \mod q\quad \text{où}\quad \vec{p} = (1\:2\:\cdots\:2^{l-1}), \]
 
	On obtient :
\[m 2^{l-1} + e_{l} \mod 2^l \]
	qui est proche de 0 si $m$ est pair et de $q/2$ sinon. On déduit de cette façon le premier bit de l'écriture en binaire de $m$ et la méthode est similaire pour complètement déduire $m$.
	
\end{subsubsection}
\begin{subsubsection}{mp\_all\_q\_decrypt: $q$ est quelconque}
	Le travail effectué ici est notamment tiré de la section 4 de \cite{EC:MicPei12}.

	En utilisant la proposition~\ref{lambda_reseau}, on constate que
\[ L = q \cdot \Lambda^\bot\left(\vec{p}\right). \]

	Il nous suffit donc de trouver une base $B$ de $\Lambda^\bot\left(\vec{p}\right)$ pour en déduire une base $q \, B^{-t}$ de $L$.

	Or, il est facile de voir que
\[ B = \begin{bmatrix} 
	2 & &&& q_0\\  
	-1 & 2 &&& q_1\\
   & -1 \\ 
	&    & \ddots & & \vdots \\
	&    &       &  2 & q_{k-2} \\
	&    &       &  -1 & q_{k-1}
\end{bmatrix}\]
convient.

On peut alors par exemple utiliser l'algorithme \textbf{nearest plane} de Baibai à partir de cette base pour déchiffrer. Notons que des bornes sur les vecteurs de la décomposition de Gram-Schmidt de cette matrice sont données dans \cite{EC:MicPei12}, ce qui peut-être intéressant, car cela est lié au domaine fondamental utilisé par l'algorithme.

\end{subsubsection}	
\end{subsection}

\begin{subsection}{Opérations homomorphes}
	On rappelle que $\vec{v}$ est de la forme $\textbf{PowersOf2}(\vec{s})$ et que donc $\textbf{Flatten}(A) \cdot \vec{v} = A \times \vec{v}$ pour tout $A$.
	
	\paragraph{}
	\textbf{MultConst}
	\flushleft
	
	\textbf{Entrée} : Cet algorithme prend en entrée les paramètres du système, un chiffré $C \in \ZZq^{N \times N}$ d'un message $\mu$ et une constante $\alpha \in \ZZq$. \\
	\textbf{Sortie} : Cet algorithme retourne un chiffré de $\alpha \cdot \mu$.\\
	\textbf{Algorithme} : On calcule $M_{\alpha} = \textbf{Flatten}(\alpha \times I_N)$ puis l'on renvoie $\textbf{Flatten}(M_{\alpha} \times C)$.
	\begin{proof}
	\begin{align*}
	\textbf{MultConst}(C, \alpha) \times \vec{v} &= M_{\alpha} \times C \times \vec{v} \\
	&= M_{\alpha} \cdot (\mu * \vec{v} + \vec{e}) \\
	&= M_{\alpha} \times \mu * \vec{v} + M_{\alpha} \times \vec{e} \\
	&= \alpha * \mu * \vec{v} + M_{\alpha} \times \vec{e}
	\end{align*}
	\end{proof}
	\textbf{Erreur} : Le chiffré à une erreur $e_2 = M_\alpha \times
		\vec{e}$. 
		\[\norm{e_2} \leq N \norm{e_1}\]
	\paragraph{}
	\textbf{Add}
	\flushleft
	
	\textbf{Entrée} : Cet algorithme prend en entrée les paramètres du système et deux chiffrés $C_1, C_2 \in \ZZq^{N \times N}$ des messages $\mu_1, \mu_2 \in \ZZq$.\\
	\textbf{Sortie} : Cet algorithme retourne un chiffré de $\mu_1 + \mu_2$.\\
	\textbf{Algorithme} : On calcule et on retourne $\textbf{Flatten}(C_1 + C_2)$.
	\begin{proof}
	\begin{align*}
	\textbf{Add}(C_1, C_2) \times \vec{v} &= (C_1 + C_2) \times \vec{v} \\
	&= (\mu_1 * \vec{v} + \vec{e_1}) + (\mu_2 * \vec{v} + \vec{e_2}) \\
	&= (\mu_1 + \mu_2) * \vec{v} + \vec{e_1} + \vec{e_2}
	\end{align*}
	\end{proof}
\textbf{Erreur} : Le chiffré à une erreur $e_3 = \vec{e_1} + \vec{e_2}$.
\[\norm{e_3} \leq \norm{e_1} + \norm{e_2}\] 
	\paragraph{}
	\textbf{Mult}
	\flushleft
	
	\textbf{Entrée} : Cet algorithme prend en entrée les paramètres du système et deux chiffrés $C_1, C_2 \in \ZZq^{N \times N}$ des messages $\mu_1, \mu_2 \in \ZZq$.\\
	\textbf{Sortie} : Cet algorithme retourne un chiffré de $\mu_1 * \mu_2$. \\
	\textbf{Algorithme} : On calcule et on retourne $\textbf{Flatten}(C_1 \times C_2)$.
	\begin{proof}
	\begin{align*}
	\textbf{Mult}(C_1, C_2) \times \vec{v} &= (C_1 \times C_2) \times \vec{v} \\
	&= C_1 \times (\mu_2 * \vec{v} + \vec{e_2}) \\
	&= \mu_2 * C_1 \times \vec{v} + C_1 \times \vec{e_2} \\
	&= \mu_2 * (\mu_1 * \vec{v} + \vec{e_1}) + C_1 \times \vec{e_2} \\
	&= (\mu_1 * \mu_2) * \vec{v} + \mu_2 * \vec{e_1} + C_1 \times \vec{e_2}
	\end{align*}
	\end{proof}
	\textbf{Erreur} : Le chiffré à une erreur $e_3 = \mu_2 * \vec{e_1} + C_1 \times \vec{e_2}$. La matrice $C_1$ étant de la forme $\textbf{Flatten}(c_1)$, elle ne contient que des 0 et des 1. 
\[\norm{e_3} \leq \mu_2 \norm{e_1} + N \norm{e_2} \]
	\paragraph{}
	\textbf{NAND}
	\flushleft
	
	\textbf{Entrée} : Cet algorithme prend en entrée les paramètres du système et deux chiffrés $C_1, C_2 \in \ZZq^{N \times N}$ des messages $\mu_1, \mu_2 \in \{ 0,1\}$.\\
	\textbf{Sortie} : Cet algorithme retourne un chiffré de $\overline{(\mu_1 \land \mu_2)} = 1 - \mu_1 * \mu_2$. \\
	\textbf{Algorithme} : On calcule et on retourne $\textbf{Flatten}(I_N - C_1 \cdot C_2)$.
	\begin{proof}
	\begin{align*}
	\textbf{NAND}(C_1, C_2) \times \vec{v} &= (I_N - C_1 \times C_2) \times \vec{v} \\
	&= \vec{v} - \textbf{Mult}(C_1, C_2)\vec{v} \\
	&= \vec{v} - (\mu_1 * \mu_2) * \vec{v} - \mu_2 * \vec{e_1} + C_1 \times \vec{e_2} \\
	&= (1 - \mu_1 * \mu_2) * \vec{v} - \mu_2 * \vec{e_1} - C_1 \times \vec{e_2}
	\end{align*}
	\end{proof}
	\textbf{Erreur} : Le chiffré à une erreur 
	$e_3 = -(\mu_2 * \vec{e_1} + C_1 \times \vec{e_2})$. On est dans 
	un contexte similaire à $\textbf{Mult}(C_1, C_2)$, mais 
	$\mu_2$ est ici égal à 0 ou 1.
\[\norm{e_3} \leq \norm{e_1} + N \norm{e_2} \leq (N+1) \max(\norm{e_1}, \norm{e_2})\]
	\end{subsection}
	\begin{subsection}{Correction du cryptosystem}
	Ici, nous considérons comme seules opérations homomorphes les portes
	NAND et comme algorithme de déchiffrement \textbf{Dec}.
	\begin{prop}
	Si la condition de longueur 
	\[q > 8nm (1 + N)^L \]
	est respectée. On peut appliquer $L$ portes NAND à un chiffré
	de $0$ ou de $1$ et le déchiffrer correctement.
	\end{prop}
	\begin{proof}
	Pour $\mu \in \{0,1\}$, $C = \text{\textbf{Encrypt}}(m)$, on a:
	\[
	C\cdot \vec{v} = \mu \vec{v} + \langle \text{BitDecomp}(R\cdot A), \vec{v} \rangle 
	\]
	On peut donc minorer ainsi son erreur:
	\begin{align*}
	\norm{\langle \text{BitDecomp}(R\cdot A), \vec{v} \rangle}
	&= \norm{\mu \vec{v} + \langle R\cdot A, \vec{s} \rangle}\\
	&= \norm{\mu \vec{v} + \langle R\cdot , A\cdot \vec{s} \rangle} \\
	&\leq  m n \qquad \text{car $R$ est constituée de $0$ ou $1$ et que}
	\: \norm{A\cdot\vec{s}} \leq n
	\end{align*}
	De plus, on peut utiliser l'inégalité indiquée dans la définition de
	NANd pour voir qu'après l'application de $i$ portes, le bruit $e_i$ du
	chiffré  $c_i$ satisfait:
	\[ \norm{e_i} \leq {(N+1)}^i  mn \]
	On voit donc qu'il suffit de montrer qu'on peut correctement déchiffrer
	$c_L$. Par la proposition~\ref{dec}, on voit qu'il faut alors avoir:
	\[ \norm{e_L} < q/8 \] 
	qui est satisfaite clairement si la condition de longueur l'est.
	\end{proof}
	\end{subsection}
\end{section}
