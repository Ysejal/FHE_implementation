\begin{section}{Notions préliminaires}
	\begin{subsection}{Différentes normes}
Pour $x\in ZZ$, on note $\bnorm{x} := \lfloor \log(x) \rfloor + 1$.

Soit $q > 0$. La valeur absolue d'un élément $x \in \ZZq$
sera par définition la valeur absolue dans $\ZZ$ de son représentatant dans
$\rrbracket - q/2, q/2 \rrbracket$. 
	La norme infinie $\norm{\vec{x}}$ d'un vecteur $\vec{x} \in \ZZq^n$ sera 
alors le maximum des valeurs absolues de ses coordonnées et la norme
$\|x\|_{1}$ la somme de des valeurs absolues de ses coordonnées.

	\end{subsection}
	\begin{subsection}{LWE et DLWE}
	Nous présentons ici les définitions du Learning with Error (LWE) dans leur version décisionnelle et calculatoire.
	
	\begin{definition}{Decisional Learning with Errors (DLWE), version décisionnelle}

	Pour un paramètre de sécurité $\lambda$, soit $n = n(\lambda),\ q = q(\lambda)$ des entiers et $\chi =
	\chi(\lambda)$ une distribution sur $\ZZ$.
	
	Le problème $\DLWE$ consiste à devoir distinguer deux distributions sur $\ZZq^{n+1}$ à partir d'un nombre
	polynomial $m = m(\lambda)$ d'échantillons qu'une des deux à produite. La première distribution crée des
	vecteurs $(\vec{a}_i,b_i) \in \ZZq^{n+1}$ uniforme. La deuxième utilise un $\vec{s} \in \ZZq^n$ tiré
	uniformément et prend pour valeurs des vecteurs $(\vec{a}_i, b_i)$ pour lesquels : \[ b_i = \langle \vec{a}_i,
	\vec{s} \rangle + e_i \] où $e_i$ est crée obtenue par $\chi$. \\ \end{definition}
	Notons qu'alors, $n = O(P(\lambda))\text{ et }\log(q) = O(P(\lambda))$ pour un polynome $P$.

	\begin{definition}{Learning with Errors (LWE)}
	
	Pour un paramètre de sécurité $\lambda$, soit $n = n(\lambda),q = q(\lambda)$ des entiers et $\chi =
	\chi(\lambda)$ une distribution sur $\ZZ$. On tire $\vec{s} \in \ZZq^n$ uniformément et on considère la
	distribution qui prend pour valeurs des vecteurs $(\vec{a}_i, b_i)$ pour lesquels :
	\[ b_i = \langle \vec{a}_i, \vec{s} \rangle + e_i \]
	où $e_i$ est crée obtenue par $\chi$.

	Le problème $\LWE$ consiste à trouver $\vec{s}$ à partir d'un nombre polynomial $m = m(\lambda)$ d'échantillons.
	\end{definition}

	Ces deux problèmes sont en fait \og équivalents \fg~. C'est assez évident de LWE vers DLWE. De plus, le Lemme 4.2
	de \cite{STOC:Regev05} montre comment réduire à DLWE à LWE sous certaines hypothèses lorsque $q$ est premier et
	$q = \mathcal{O}(\text{poly}(n))$. Le théorème 3.1 de \cite{EPRINT:MicPei11} montre la même chose mais lorsque
	$q$ est un produit de premiers $p_i \in \mathcal{O}(\text{poly}(n))$, comme ce sera le cas lorsque nous
	considèrerons $q = 2^k$.

	Voyons par exemple le cas (plus facile) où $q$ est premier :

	\begin{prop}{DWLE vers LWE}

	Soit $n \geqslant 1$ un entier, $2 \leqslant q \leqslant \text{poly}(n)$ un nombre premier et $\chi$ une
	distribution sur $\ZZq$. Supposons avoir accès à un automate $\mathcal{W}$ qui accepte avec une probabilité
	exponentiellement proche de 1 les distributions $A_{s, \xi}$ et rejete avec une probabilité exponentiellement
	proche de 1 la distributions uniforme $U$.
	
	Il existe alors un automate $\mathcal{V}$ qui, étant donné des échantillons de $\mathcal{A}_{s,\chi}$ pour un
	certain $s$, retrouve $s$ avec une probabilité exponentiellement proche de 1.
	\end{prop}
	\begin{proof}
	Nous indiquons ici la démonstration faite dans \cite{STOC:Regev05}.
	
	L'automate $W'$ va trouver $s$ coordonnée par coordonnée. Montrons comment $W'$ obtient la première coordonnée $s_1$.
	
	Pour $k \in \ZZq$, on considère la fonction :
	\[f_{k,1}: (a,b) \mapsto (a + (l, 0, ..., 0), b + l \cdot k) \]
	avec $l\in \ZZq$ échantilloné uniformément sur $\ZZq$.
	
	$f_{k,1}$ appliquée à un échantillon uniforme donne un échantillon uniforme tandis qu'appliquée à un échantillon
	de $A_{s, \chi}$, elle donne un échantillon de $A_{s, \chi}$ si $k = s_1$, et uniforme sinon.
	
	On peut faire une recherche exhaustive sur les $k \in \ZZq$ jusqu'à en trouver un accepté par $W$, qui sera le
	bon avec une probabilité exponentiellement proche de 1.
	
	Cela se fait en temps polynomial car $q < \text{poly}(n)$ et $f_{k,1}$ s'execute en temps polynomial.
	
	On peut effectuer la même chose avec la fonction
	\[f_{k,i}: (a,b) \mapsto (a + (0, 0, ..., l, 0, ..., 0), b + l \cdot k) \]
	avec le $l$ ajouté à $a$ en $i$ème position $\forall i$.
	
	On retrouve ainsi $s$ avec $n$ calculs polynomiaux en $n$, ce qui reste évidemment polynomial en $n$.
	
	La probabilité de se tromper est $n$ fois quelque chose d'exponentiellement proche de 0 et reste donc
	exponentiellement proche de 0.
	\end{proof}

	Pour analyser la sécurité du cryptosystème, nous utiliserons le problème DLWE. Comme l'indique le théorème 1 de
	\cite{C:GenSahWat13}, il est possible de réduire le problème LWE à des problèmes sur des réseaux.

	Indiquons ici de façon informelle comment passer du problème LWE à un problème de type SVP (short vector
	problem). Tout d'abord, nous aurons besoin d'exprimer LWE sous une forme matricielle :

	\begin{definition}{versions matricielles de DLWE et LWE}

	En prenant les paramètres de la précédente définition, le problème $\DLWE$ consiste à décider si une matrice $A \in \ZZq^{m \times (n+1)}$ est uniforme ou bien s'il existe un vecteur $\vec{v} = (1\quad -\vec{s})$ tel que $A \cdot \vec{v} \in \ZZq^{m}$ est créé à partir de $\chi^m$. Autrement dit, avec les notations de la formulation classique de LWE, si les lignes de $A$ sont de la forme $(b_i, \vec{a}_i)$.
	
	Le problème $\LWE$ consiste lui à trouver $\vec{v}$ à partir de $A$.
	\end{definition}

	Nous allons ici considérer le problème LWE calculatoire, dans lequel il faut trouver le vecteur $\vec{v}$ tel que :
	\[ A\cdot \vec{v} = \vec{e} \mod q \]
	où les coordonnées de $\vec{e}$ sont créées par $\chi$.

	De façon équivalente, il faut trouver un vecteur $(*\quad\vec{v})$ tel que :
	\[ \begin{bmatrix}q & A \\ 0 &1 \end{bmatrix}\cdot
	   \begin{bmatrix}* \\ \vec{v} \end{bmatrix} =
	   \begin{bmatrix} \vec{e} \\ \vec{v} \end{bmatrix} \]
	Si la distribution $\chi$ créé de petites valeurs, on voit qu'on a alors trouvé un \og petit \fg~ vecteur du réseau engendré par les colonnes de 
	\[ \begin{bmatrix}q & A \\ 0 &1 \end{bmatrix} \]

	\textbf{Choix de paramètres pour DLWE:}

	Interessons nous maintenant à la façon de choisir des paramètres
	pour que le problème $DLWE$ soit difficile. Nous utilisons pour cela 
	les remarques faites dans \cite{halevi}.

	Déjà, il impose que la distribution $\chi$ soit \og concentrée sur de
	petites valeurs \fg~. Plus précisément, il pose l'hypothèse suivante: 

	\begin{hyp}{Hypothèse sur la probabilité $\chi$}\\ \label{hyp:proba}
	il doit exister $\alpha = \alpha(n) << 1$ tel que:
	\[ n \mapsto \PP [x \leftarrow \chi : |x| > \alpha q]\] 
	soit négligeable.
	\end{hyp}

	il faut donc $\alpha q > 1$ pour que $\chi$ puisse générer autre chose 
	que des zéros. 
	On considère alors deux hypothèses de difficulté 
	concernant DLWE. 

	\begin{hyp}{hypothèse pour le leveled GSW} \\
	\label{hyp_dlwe}
	Il existe un $0 < \epsilon < 1$ tel que le problème DLWE soit
	soit difficile pour 
	\[ q \approx 2^{n^\epsilon}\quad \alpha q = n\quad \text{$m$ polynomial en $n$}\]
	\end{hyp}

	Pour le GSW avec bootstrapping, nous considèrerons plutôt l'hypothèse
	suivante:
	\begin{hyp}{hypothèse pour le GSW avec bootstrapping} \\
	\label{hyp_dlwe_boot}
	Le problème DLWE est difficile pour 
	\[ q \approx 2^\text{polylog$(n)$}\quad \alpha q = n\quad \text{$m$ polynomial en $n$}\]
	\end{hyp}


	\end{subsection}
	\begin{subsection}{Réseaux euclidiens}
	Nous rappelons ici quelques résultats sur les réseaux euclidiens, tels qu'énoncés dans \cite{EC:MicPei12}. Ils nous serons utiles pour définir les gaussiennes discrètes ainsi que pour comprendre un des algorithme de déchiffrement du cryptosystème GSW.

	Tous les réseaux considérés ici sont de rang plein, autrement dit, si $\L \subset \RR^n$, alors $L$ est de dimension $n$.

	\begin{definition}
	Soit $L\subset \RR^n$ un réseau. Le dual de $L$ est défini comme étant :
	\[ L^* = \{ v \in \RR^n \::\: \langle x, v \rangle \in \ZZ
	   \:\text{ pour tout } x\in L\} \]
	\end{definition}
	\begin{prop} 
	Soit $L \subset \RR^n$ un réseau euclidien. Soit $B$ est une base de $L$.
	
	$B^{-t}$ est une base de $L^*$.
	\end{prop}

	Pour $q$ un entier et $A \in \ZZ^{n\times m}$, on pose :
		\[\Lambda^\bot(A) = \{ z \in \ZZ^m\: :\: A z  = 0 \mod q \}\] 
	\[\Lambda(A^t) = \{ z \in \ZZ^m\: : \: \exists s\in \ZZq^n, z = A^t s \mod q\}\]

		\begin{prop} \label{lambda_reseau}
	Conservant les notations précédentes, 
	\[q \cdot {\Lambda^\bot(A)}^* =  \Lambda(A^t)\] 
	\end{prop}
	\end{subsection}
	
	\begin{subsection}{La gaussienne discrète}
	Très souvent, la distribution $\chi$ choisi pour avoir des paramètres sécurités pour le problème LWE est une gaussienne discrète. Nous nous proposons ici d'en indiquer la définition, ainsi que d'en indiquer certaines propriétés.
Notamment, la propriété~\ref{gaussienne_alpha} montre qu'il suffit 
de prendre $\chi = D^q_s$ avec un $s > 1.5$ constant pour 
qu'avec les valeurs de $\alpha$ spécifiées dans les hypothèses~\ref{hyp_dlwe} et
\ref{hyp_dlwe_boot}, 
l'hypothèse~\ref{hyp:proba} est satisfaite.

	Rappellons aussi qu'une famille $\{\chi_n\}_n$ de distributions est dite $B$-bornée pour une borne $B = B(n)$ si la fonction suivante est négligeable :
		\[n \mapsto \PP\left(\chi_n > B(n)\right) \]

	\paragraph{}
	Nous reprenons ici les notations de \cite{STOC:GenPeiVai08}.

	Soit un entier $n > 0$  et $s > 0$. On définit la densité gaussienne sur $\RR^n$ comme la fonction qui à $x\in\RR^n$ attribue:
	\[\rho_{s,c}(x) = e^{\phi * {\frac{||x-c||}{s}}^2} \]

	Puis, pour un réseau $\Lambda \in \RR^n$, nous définissons la
	gaussienne discrète $D_{\Lambda,s,c}$ comme la distribution de support $\Lambda$ de loi de probabilité: 	
	\[ D_{\Lambda, s, c}(x) = \frac{\rho_{s,c}(x)}{\sum_{l\in \Lambda}\rho_{s,c}(l)}\]

	Enfin, pour un entier $q > 0$, nous définissons la gaussienne discrète
	$D^q_{s}$ modulo un entier $q > 0$ comme la composition la fonction qui a $x \in \ZZq$ attribue 
		\[ D_{\ZZ, s, 0}(\pi^{-1}(x)) \]
	où $\pi$ est la projection $\ZZ \rightarrow \ZZq$.

La définition et la proposition suivante (Le lemme 4.2 de \cite{STOC:GenPeiVai08}) permettent de trouver une borne à une famille de gaussienne.
\begin{definition}
Pour un réseau $L$ de dimension $n$ et un réel $\epsilon > 0$, le paramètre
$\eta_\epsilon(L)$, dit smoothing parameter, est le plus petit réel $s>0$ tel que 
	\[\rho_{1/s}(L^* \setminus \{0\}) \leqslant \epsilon\]
\end{definition}
	\begin{prop}
	\label{gaussienne}
	Pour tout $\epsilon > 0$ , $s \geqslant \eta_{\epsilon}(\ZZ)$ et pour tout $t>0$ :
	\[ \PP\left(x \leftarrow D_{\ZZ, s,c}\: :\: |x-c| \geq t\cdot s\right) \leqslant 2 e^{-\pi t^2}	\cdot \frac{1+\epsilon}{1-\epsilon} \]
	% Notamment, pour $0 < \epsilon  < 1/2$ et $t \geqslant \omega(\sqrt{\log(n)})$, cette probabilité est négligeable.
	\end{prop}

	L'article \cite{cryptoeprint:2018:786} indique une borne intéressante sur le smoothing parameter:
\begin{prop}
Pour $\epsilon < 0.086435$, on a la borne:
\[\eta_{\epsilon}(\ZZ)  \leq
\sqrt{\frac{\ln\left(\epsilon/44 + 2/\epsilon\right)}{\pi}} \]
\end{prop}

\begin{prop}
\label{gaussienne_alpha}
Pour $s \geq 1.5$ et $\alpha = \alpha(\lambda)$, si il existe une constante $\tau$
telle que $\alpha \geq \tau n/q$, alors:
\[\lambda \rightarrow  \PP\left(x\leftarrow D^q_s\: : \: |x|\geq q \alpha \right)\]	
est négligeable et $D^q_s$ satisfait donc l'hypothèse~\ref{hyp:proba}.
\end{prop}
\begin{proof}
On veut appliquer la proposition~\ref{gaussienne}.
Notons d'abord que la borne précédemment donnée sur 
permet de dire que pour $epsilon = 0.08$ et $s > 1.5$, on a bien:
\[ s > \eta_{\epsilon}(\ZZ) \]

De plus, on a:
\begin{align*}
\PP\left(x\leftarrow D^q_s\: : \: |x|\geq q \alpha \right) =
\PP\left(x\leftarrow D^q_s\: : \: |x|\geq s \left(\frac{q \alpha}{s}\right)  \right)
\end{align*}
On en déduit qu'on peut appliquer la proposition précédente avec $t = \frac{q
\alpha}{s} $.

On en déduit que pour que la probabilité soit négligeable, il suffit qu'il
existe un $u > 0$ tel que:
\[q \alpha / s \geq u n\]
Or, on constate que  $u = s\tau$ convient.
\end{proof}

	\end{subsection} % gaussiene discrete
\end{section}
