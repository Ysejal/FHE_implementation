\begin{section}{Notions préliminaires}
	\begin{subsection}{LWE et DLWE}
	Nous présentons ici les définitions du Learning with Error
	(LWE) dans leur version décisionnelle et calculatoire.
	
	\begin{definition}{Decisional learning with Errors (DLWE), version décisionnelle}
	Pour un paramètre de sécurité $\lambda$, soit $n = n(\lambda),q = q(\lambda)$ 
	des entiers et $\chi = \chi(\lambda)$ une
	distribution sur $\ZZ$.
	Le problème $\DLWE$ consiste à devoir distinguer 
	deux distributions sur $\ZZq^{n+1}$ à partir d'un nombre polynomial
	de $m = m(\lambda)$ d'échantillons qu'une des deux à produite.
	La première distribution crée des vecteurs $(\vec{a}_i,b_i) \in
	\ZZq^{n+1}$ uniforme.
	La deuxième utilise un $\vec{s} \in \ZZq^n$ tiré uniformément et 
	prend pour valeurs des vecteurs $(\vec{a}_i, b_i)$ pour lesquels:
	où $e_i$ est crée obtenue par $\chi$.
	\end{definition}

	\begin{definition}{learning with Errors (LWE)}
	Pour un paramètre de sécurité $\lambda$, soit $n = n(\lambda),q = q(\lambda)$ 
	des entiers et $\chi = \chi(\lambda)$ une
	distribution sur $\ZZ$. On tire $\vec{s} \in \ZZq^n$ uniformément 
	et on considère la distribution qui  
	prend pour valeurs des vecteurs $(\vec{a}_i, b_i)$ pour lesquels:
		\[ b_i = \langle \vec{a}_i, \vec{s} \rangle + e_i \]
	où $e_i$ est crée obtenue par $\chi$.

	Le problème $\LWE$ consiste à trouver $\vec{s}$ à partir d'un nombre
	polynomial $m = m(\lambda)$ d'échantillons.
	\end{definition}

	Ces deux problèmes sont en fait \og équivalents \fg. Cela semble facile
		de LWE vers DLWE. De plus,
	le Lemme 4.2 de \cite{STOC:Regev05} montre comment réduire à
	DLWE à LWE sous certaines hypothèses lorsque $q$ est premier, et 
	le théorème 3.1 de \cite{EPRINT:MicPei11} lorsque $q$ est un produit 
	de petit premier, comme ce sera le cas lorsque nous considèrerons 
	$q = 2^k$.

	
	Pour analyser la sécurité du cryptosystème, nous utiliserons le
	problème $DLWE$. 
	Comme l'indique le théorème 1 de \cite{C:GenSahWat13}, il est
	possible de réduire le problème LWE à des problèmes sur des réseaux.


	Indiquons ici de façon informelle comment passer du problème LWE à un 
	problème de type SVP (short vector problem).
	Tout d'abord, nous aurons besoin d'exprimer LWE sous une forme
	matricielle:

	\begin{definition}{versions matricielles de DLWE et LWE}
	En prenant les paramètres de la précédente définition

	Le problème $\DLWE$ concisite à décider 
	si une matrice $A \in \ZZq^{m \times (n+1)}$ 
	est uniforme ou bien si il existe un vecteur $\vec{v} = (1\quad
	-\vec{s})$ tel que $A \cdot \vec{v} \in \ZZq^{m}$ est 
	crée à partir de $\chi^m$. Autrement dit, avec les notations de 
	la formulation classique de LWE, si les lignes de $A$ sont de la forme
		$(b_i, \vec{a}_i)$. 

	Le problème $\LWE$ consiste lui à trouver $\vec{v}$ à partir de $A$.
	\end{definition}



	Nous allons ici considérer le problème LWE calculatoire, dans lequel il
	faut trouver le vecteur $\vec{v}$ tel que:

	\[ A\cdot \vec{v} = \vec{e} \mod q \]
	où les coordonnées de $\vec{e}$ sont créées par $\chi$.

	De façon équivalente, il faut trouver un vecteur $(*\quad\vec{v})$ tel
	que:
	\[ \begin{bmatrix}q & A \\ 0 &1 \end{bmatrix}\cdot
	   \begin{bmatrix}* \\ \vec{v} \end{bmatrix} =
	   \begin{bmatrix} \vec{e} \\ \vec{v} \end{bmatrix} \]
	Si la distribution $\chi$ créé de petites valeurs, on voit qu'on à
	alors trouvé un \og petit \fg vecteur du réseau engendré par les colonnes de 
	\[ \begin{bmatrix}q & A \\ 0 &1 \end{bmatrix} \]
	\end{subsection}
	\begin{subsection}{La gaussienne discrète}
	Nous définitions la gaussienne discrète
	$D_\alpha$ comme l'unique distribution sur $\ZZ$ dont la loi de probabilité 
	est proportionnelle à :
	\[ f(k) = e^{-\pi {\left(\frac{k}{\alpha}\right)}^2}\]

	De même, pour $B > 0$, 
	Nous définitions la gaussienne discrète $B$-bornée $D_{\alpha,B}$ comme l'unique distribution sur $\ZZ$ dont la loi de probabilité 
	est proportionnelle à :
	\[ f(k) = \begin{cases}e^{-\pi {\left(\frac{k}{\alpha}\right)}^2}
	\quad\text{si}\:\: |k| < B \\ 0 \quad \text{sinon}\end{cases}\]

	Pour un entier $q > 0$,  
	Nous définissions enfin la gaussienne discrète $D^q_{\alpha}$ (resp. discrète
	$B$-bornée $D^q_{\alpha,B},\: B < q$) modulo un entier $q > 0$ comme la
	composition de $D_\alpha$ 
	(resp. $D_{\alpha,B}$) avec la projection sur $\ZZq$.
	\end{subsection}
\end{section}
