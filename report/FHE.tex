\begin{section}{FHE, SFHE et bootstrapping}

Nous indiquons ici les définitions de base d'un FHE tirées de 
l'exposition faite dans~\cite{halevi} que nous conseillons 
de lire pour avoir plus de détails.

Notons que le cryptosystème que nous allons étudier a pour ensemble de clairs $\ZZq$, mais que ces définitions seront faites pour un cryptosystème ne permettant de ne chiffrer que 0 ou 1.
Cela peut sembler réducteur, mais nous ne prouverons que le  cryptosysteme
GSW peut-être adapté en FHE et LHE\footnote{voir les défintions suivantes} qu'en restreignant 
les clairs à l'ensemble $\{0, 1\}$.

\begin{definition}{Cryptosystème homomorphe}

Un cryptosystème homomorphe est constitué de 4 fonctions :
\[ \text{\textbf{HE}} = (\Keygen, \Enc, \Dec, \Eval)\]
où l'espace des clairs est $\{0,1\}$ et l'espace des chiffrés est un ensemble $C$.
Plus précisément:
\begin{itemize}
\item $(\pk, \sk) \leftarrow \Keygen(1^\lambda)$, $\lambda$ étant le paramètre de sécurité, retourne une clé publique, une clé d'évaluation et une clé secrète.
CLÉ D'ÉVALUATION ? IL N'Y A QUE DEUX ÉLÉMENTS RETOURNÉS. WTF ?
\item $c \leftarrow \Enc_{\pk}(\mu)$ prend une clé publique $\pk$, un clair $\mu$ et retourne un chiffré $c\in C$
\item $\mu \leftarrow \Dec_{\sk}(c)$ prend une clé secrète $\sk$, un chiffré $c \in C$ et retourne un clair $\mu$.
\item $\vec{c}_f \leftarrow \Eval_\pk(\Pi,\vec{c})$ prend une clé publique
$\pk$, un circuit  $\Pi$, un vecteur de chiffrés $\vec{c}$ et retourne un
vecteur de chiffrés $\vec{c}_f$ contenant un chiffré chaque sortie du circuit $\Pi$.
OK, ALORS LÀ, ÇA NE VEUX PLUS RIEN DIRE. ÇA RETOURNE QUOI ? WTF ?
\end{itemize}
\end{definition}

\begin{definition}{Correction}

Soit \textbf{E} = $(\Keygen, \Enc, \Dec, \Eval)$ un cryptosystème homomorphe
et  $\mathcal{C} = \{ \mathcal{C}_\tau\}_{\tau \in \mathbb{N}}$ une famille de
circuits.  On dira que \textbf{E} est correct pour $\mathcal{C}$ si il déchiffre avec succès les messages qui viennent d'être chiffrés 
ainsi que ceux qui ont été évalués sur un circuit de $\mathcal{C}$. Voir \cite{halevi} pour les
formules détaillées.
\end{definition}

\begin{definition}{Compacité}

Un cryptosystème homomorphe \textbf{E} = $(\Keygen, \Enc, \Dec, \Eval)$ 
est compact si il existe un polynome $B$ tel que pour tout $\lambda, \tau
\in \mathbb{N}$, tout circuit $\Pi$ à $t$ entrées et une seule sortie, et un
chiffré $b = (b_1,\cdots,b_t) \in \{0,1\}^t$, on a:
\[ 
\PP \left[ |c'| \leqslant B(\lambda): (\!\sk,\pk) \leftarrow \Keygen(1^\lambda,
1^\tau), \vec{c} \leftarrow \Enc_\pk(\vec{b}), \vec{c} \leftarrow 
\Eval_\pk(\Pi, \vec{c}) \right] = 1.
\]
\end{definition}

\begin{definition}{FHE, LHE et SWHE}

Soit \textbf{E} un cryptosystème homomorphe.
\begin{itemize}
\item On dit que c'est un fully homomorphic encryption (abrégé en FHE) si il est correct pour
une famille $\mathcal{C}$ telle que $\mathcal{C}_1$ contient déjà tous les
circuits booléens. 
\item On dit que c'est un  leveled homomorphic encryption (abrégé en LHE)  si il est correct pour une famille
$\mathcal{C}$ telle que pour tout $\tau$, $\mathcal{C}_\tau$ contient les circuits booléens de profondeur plus petite ou
égale à $\tau$. Plus généralement, l'idée est de pouvoir prévoir, pour tout $L$, les paramètres à utiliser afin de
pouvoir évaluer avec succès les circuits d'une profondeur $L$ 
\item On dit, informellement, qu'il s'agit d'un somewhat homomorphic encryption (abrégé en SWHE) si il est correct pour
une famille $\mathcal{C}$ de circuits tels que la complexité des circuits de $\mathcal{C}_\tau$ grandis avec $\tau$.
\end{itemize}
\end{definition}

Nous allons maintenant définir le bootstrapping en introduisant pour cela
quelques notations.

\begin{itemize}
\item Pour $\lambda, \tau \in \mathbb{N}$, on note
$\mathcal{CT}_{\text{\textbf{E}}}(\lambda,\tau)$ l'union des chiffrés de messages 
ainsi que des évaluations de chiffrés de messages par des circuits de 
$\mathcal{C}_\tau$, les formules détaillées étant donnés dans \cite{halevi};
\item On considère pour tout chiffré $c\in\mathcal{CT}_{\text{\textbf{E}}}(\lambda,\tau)$ 
le circuit $D_c(\sk) = \Dec_\sk(c)$ ayant pour entrée $\sk$ et appliquant 
l'algorithme de déchiffrement de $c$ avec la clé $\sk$ (notons bien que $c$
n'est pas une entrée du circuit). On crée aussi le circuit suivant, avec pour
unique entrée \sk, appelé circuit de déchiffrement augmenté de $c_1$ et $c_2$
\[ D^*_{c_1,c_2}(sk) := \text{NAND}(D_{c_1}(\sk), D_{c_2}(\sk)).\]
\end{itemize}

\begin{definition}{Bootstrappable encryption}

Un cryptosystème homomorphe
\textbf{E} = $(\Keygen, \Enc, \Dec, \Eval)$
est bootstrappable si il peut évaluer homomorphiquement tous les circuits 
de chiffrement augmentés. Autrement dit, si il existe une fonction $\tau$
bornée par un polynôme tel que pour tous $\lambda \in \mathbb{N}$ et tous
chiffrés  $c_1, c_2 \in \mathcal{CT}_{\text{\textbf{E}}}(\lambda, \tau(\lambda))$, on a
$D^*_{c_1,c_2} \in \mathcal{C}_{\tau(\lambda)}$.
\end{definition}

\begin{definition}{weak circular security}

\label{def:circular}
Soit \textbf{E} = $(\Keygen, \Enc, \Dec)$ un cryptosystème et $\mathcal{A}$ un
algorithme polynomial probabiliste. 
On considère l'expérience suivante, 
\begin{itemize}
\item $1^\tau \leftarrow \mathcal{A}(1^\lambda)$.
\item $(\sk, \pk) \leftarrow \Keygen(1^\lambda, 1^\tau)$.
\item On chiffre chaque bit de la clé secrète et on obtient un vecteur de chiffrés $\vec{c^*} \leftarrow \Enc_\pk{s_k}$.
\item Un bit $b\in \{0,1\}$ est tiré uniformément.
\item On crée $c \leftarrow \Enc_\pk(b)$ .
\item $b^* \leftarrow \mathcal{A}(\pk, \vec{c^*}, c)$.
\item L'expérience est réussie si $b = b^*$ et ratée sinon.
\end{itemize}
Notant $p(\lambda, \mathcal{A})$ la probabilité de succès de cette expérience, on dit
que $\mathcal{A}$ a un avantage de sécurité circulaire sur \textbf{E} si:
\[\lambda \mapsto |p(\lambda, \mathcal{A}) - \frac{1}{2}| \]
est négligeable.
\end{definition}

\begin{thm}
Tout cryptosysteme homomorphe bootstrappable avec sécurité circulaire peut être transformé en un 
cryptosystème fully homomorphe compact.
\end{thm}
\end{section}
