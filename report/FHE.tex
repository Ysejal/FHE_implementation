% \begin{section}{FHE, SFHE et bootstrapping}


Nous indiquons ici les définitions de base d'un FHE en adaptant
l'exposition faite dans~\cite{halevi}.

\begin{definition}
Un cryptosystème homomorphe est constitué de $4$ fonctions:
\[ \text{\textbf{HE}} = (\Keygen, \Enc, \Dec, \Eval)\]
où $M$ est  l'espace des clairs, et un espace $C$ de chiffré. Plus précisément:
\begin{itemize}
\item $(\pk, \evk, \sk) \leftarrow \Keygen(1^\kappa)$, $\kappa$ étant le paramètre de sécurité, retourne une clé publique, une clé d'évaluation et une clé secrète.
\item $c \leftarrow \Enc_{\pk}(\mu)$ prend une clé publique $\pk$, un clair $\mu \in R$ et retourne un chiffré $c\in C$
\item $\mu \leftarrow \Dec_{\sk}(c)$ prend une clé secrete $\sk$, un chiffré $c \in C$ et retourne un clair $\mu \in M$.
\item $c_f \leftarrow \Eval_\evk(f, c_1,\cdots,c_l)$ prend une clé d'évaluation
$\evk$, une fonction $f: R^l \rightarrow R$, des chiffrés $(c_1, \cdots, c_l)$ et retourne un chiffré $c_f$.
\end{itemize}
\begin{definition}


% \begin{definition}{Cryptosystème $\mathcal{C}$-homomorphe}
% On considère une famille d'ensembles fonctions $\mathcal{C} = \{\mathcal{C}_{k} \}_{k\in \mathbb{N}}$. \textbf{HE} est dit 
% $\mathcal{C}$-homomorphe  si pour toute fonction $f_k \in \mathcal{C}_k$, et
% pour toutes entrées $\mu_1, \cdots \mu_l \in R$ de la fonction $f_k$:
% \[ \PP\left(\Dec_\sk\left(\Eval_\evk(f,c_1,\cdots,c_l)\right) \neq f(\mu_1,\cdots,\mu_l)  \right) \]
% est négligeable en $k$, pour $(\pk, \evk, \sk) \leftarrow \Keygen(1^\kappa)$ et $c_i \leftarrow \Enc_\pk(\mu_i)$.
% \end{definition}


% % \begin{definition}{Compacité}
% % Un cryptosystème homomorphe \textbf{HE} est compact si il existe un polynome $s
% % = s(\kappa)$ tel que la longueur de la sortie de $\Eval_\evk(f,c_1,\cdots,c_l)$ est majorée
% % par $s(k)$ pour toute fonction $f\i
% % \end{definition}

% \begin{definition}
% Soit $(M, +, x, 0, 1)$ un anneau. On appelle circuit algébrique 
% l'ensemble $\mathcal{A}_M$ de fonctions stable par composition et
% contenant:
% \begin{itemize}
% \item la fonction somme $\text{\textbf{Add}}: M \times M \rightarrow M$;
% \item la fonction produit $\text{\textbf{Mul}}: M \times M \rightarrow M$;
% \item Pour tout $\alpha \in M$, la fonction de multiplication par un scalaire
% $\text{\textbf{Scal}}_\alpha: M \rightarrow M$;
% \end{itemize}
% La profondeur d'un circuit algébrique est la profondeur obtenue
% en décomposant le circuit sous la forme d'un arbre.
% \end{definition}

% % \begin{definition}{Cryptosystème pleinement homomorphe (FHE)}
% % Un cryptosystème est pleinement homomorphe si son ensemble de clairs $M$ est
% % muni d'une structure d'anneau et si il est $\mathcal{A}_M$-homomorphe.
% % \end{definition}
% % \begin{definition}{Cryptosystème pleinement homomorphe (FHE)}
% % Un cryptosystème est homomorphe pour la profondeur $h = h(\kappa)$$ si son ensemble de clairs $M$ est
% % muni d'une structure d'anneau et si il est $\mathcal{C}=\{\mathcal{C}_k\}$-homomorphe.
% % où $\mathcal{C}_k$ sont les circuits algébriques de profondeur au plus $h(k)$.
% % Notons que si on applique homomorphiquement un circuit de profondeur $p < h(k)$
% % à $c = \Enc_{\pk}(\mu)$, on peut donc encore lui appliquer un circuit de
% % profondeur $h(k) - p$.
% % \end{definition}

% \begin{definition}{leveled fully homomorphic encryption}
% Un leveled fully homomorphic encryption est un cryptosystème homomorphe pour 
% lequel $\Keygen$ contient une entrée supplémentaire $1^L$, et le schéma
% résultant est homomorphe pour les circuits algébriques de profondeur $L$.
% $\text{\textbf{HE}} = (\Keygen, \Enc, \Dec, \Eval)$
% \end{definition}

% \begin{subsection}{SFHE et bootstrapping}

% \begin{definition}{leveled fully homomorphic encryption}
% Un somewhat fully homomorphic encryption est un cryptosystème homomorphe
% pour une classe $\mathcal{C}$ de fonctions tel que $\mathcal{C}_k$
% contient $\text{\textbf{Add}} \circ \Dec_\sk$  $\text{\textbf{Mult}} \circ \Dec_\sk$ 
% $\text{\textbf{Scal}}_\alpha \circ \Dec_\sk$ et peut encore effectuer 

% lequel $\Keygen$ contient une entrée supplémentaire $1^L$, et le schéma

% $\text{\textbf{HE}} = (\Keygen, \Enc, \Dec, \Eval)$
% \end{definition}

% La plupart des FHE sont construits à partir de ce que l'on appelle un Somewhat Fully-Homomorphic Encryption (SFHE).
% Il s'agit d'un cryptosystème homomorphe pour les circuits algébriques 
% d'une certaine profondeur $h$ auquel on rajoute certaines propriété.

% L'idée est qu'appliquer homomorphiquement \textbf{Add}, \textbf{Mul} ou
% \textbf{Scal} sur des chiffrées crée un \og bruit \fg, qui lorsqu'il est trop
% grand, empêche un déchiffrement correct et limite donc la profondeur des
% circuits algébriques utilisables.

% On demande toutefois deux conditions supplémentaires:
% \begin{itemize}
% \item $\Dec_\sk$ peut s'exprimer comme un circuit algébrique appartenant à  $\mathcal{C}$;
% \item le bruit crée par ce circuit est 
% \end{itemize}


% \begin{definition}{Cryptosystème avec amorçage (bootstrappable)}
% $\text{\textbf{HE}} = (\Keygen, \Enc, \Dec, \Eval)$

% \end{definition}



% \end{section}
