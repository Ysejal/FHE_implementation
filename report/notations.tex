\begin{section}{Notations}
Soit $q > 0$. La valeur absolue d'un élément $x \in \ZZq$
sera par définition la valeur absolue dans $\ZZ$ de son représentatant dans
$\rrbracket - q/2, q/2 \rrbracket$. 
	La norme infinie $\norm{\vec{x}}$ d'un vecteur $\vec{x} \in \ZZq^n$ sera 
alors le maximum des valeurs absolues de ses coordonnées.
\end{section}
