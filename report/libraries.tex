\begin{section}{Des librairies pour du FHE}
Plusieurs librairies open-sources implémentant divers FHE sont disponibles. 
On peut notamment en trouver une liste sur HomomorphicEncryption.org \cite{homencrypt.org}, 
qui se décrit comme \og an open consortium of industry, government and academia to 
standardize homomorphic encryption \fg.

Nous proposons ici d'en évoquer deux:
\begin{itemize}
\item The Simple Encrypted Arithmetic Library (SEAL) \cite{seal}, dont nous avons tiré des paramètres  
\og réalistes \fg\footnote{Pas forcément pour nos machines et avec notre implémentation}
sécurisés et autorisant une profondeur de NAND non null (même si irréaliste: seulement 3);
\item The Gate Bootstrapping API \cite{TFHE} qui implémente une variation du cryptosystème GSW;
\end{itemize}

\begin{subsection}{La librarie SEAL}
Acronyme de Simple Encrypted Arithmetic Library, SEAL (voir \cite{seal}) 
est une librairie écrite par le \og cryptography research group \fg~ de Microsoft, en C++ sous 
licence MIT. Elle se propose d'implémenter deux FHE de seconde génération: 
BVS \cite{EPRINT:FanVer12} et CKKS \cite{AC:CKKS17}.

Son installation est facile\footnote{Sur Linux debian 4.9.0-8-amd64, nous avons dû
utilister les backports debians pour avoir une version de cmake suffisament récente}
et il est directement possible de compiler un executable permettant de tester 
diverses fonctionnalitées de la librarie. De plus, la documentation
\cite{seal_manual_231},
malheureusement non à jour, indique quelques points théoriques autant du point 
de vue mathématique que des choix de représentation des données.
\end{subsection}

\begin{subsection}{The Gate Bootstrapping API}
Notre présentation s'appuie sur celle donnée dans la page officielle (voir \cite{TFHE})
 qui est claire et bien documentée.

l'API Gate Bootstrapping est une librairie open source utilisable en C, C++ et 
s'appuyant notamment sur des travaux de I. Chillotti, N. Gama, M. Georgieve et M. Izabachène 
(voir \cite{cryptoeprint:2017:430} et  \cite{cryptoeprint:2016:870}). 

Elle utilise une version modifiée du cryptosysteme GSW (\cite{C:GenSahWat13})
étudié dans notre rapport, et permettant aussi bien du LHE que du FHE. C'est
pourquoi nous allons parler plus en détail de celle-ci.

Ses performances sont interessantes; il est notamment indiqué dans la sous-section
4.2 de \cite{cryptoeprint:2016:870} que pour un ordinateur 64-bit simple coeur 
(i7-4930MX) cadencé à 3.00GHz, le bootstrapping se fait en un temps moyen de 52ms.
de clée de bootstrapping d'environ 24MO.

Pour arriver à de tels résultats, de nombreuses modifications et optimisations dans le codes 
ont été faites. Notamment, le problème sur lequel s'appuie le cryptosystème n'est plus 
LWE mais TFHE, présenté dans les librairies suscitées.

\begin{subsubsection}{Un exemple simple}
En plus d'une présentation de leur API, leur site de présentation
contient un tutorial sous forme de 3 fichiers de codes simples 
permettant de simuler une \og communication chiffrée\fg~ entre Alice et le cloud:
\begin{itemize}
\item Alice génère des clés, chiffre des données et les envoies ainsi
	que la clé de bootstrapping au cloud;
\item Le cloud applique homomorphiquement une fonction, le minimum
	entre deux nombres, aux données et les renvoie à Alice;
\item Alice déchiffre le résultat;
\end{itemize}
Afin de manipuler la librairie, nous avons \og mis en forme \fg~ ces fichiers
en y ajoutant quelques modifications. Le tout est situé dans 
\path{using_tfhe_library} et il suffit de faire \path{make} 
pour compiler l'executable, sous couvert d'avoir la librairie tfhe installée.
	
Les fichiers sources ainsi que les hearders contiennent normalement assez de
commentaire pour être lisibles. Nous proposons donc ici de résumer brièvement 
le rôle de chaque fichier source:
\begin{itemize}
\item \path{alice.c} contient des fonctions permettant de générer
clés, chiffrer et déchiffrer;
\item \path{homomorphic_functions.c} contient deux exemples de fonctions 
appliquables homomorphiquement: le minimum de deux nombres (déjà 
présent dans le tutorial) et leur somme;
\item \path{cloud.c} contient une fonction permetant d'appliquer 
homomorphiquement une des fonctions de \path{homomorphic_functions.c}
sur des chiffrés puis d'enregistrer le résultat;
\item enfin, \path{example_communication.c} utilise les fichiers précédents
	pour simuler une communication entre Alice et le cloud.
\end{itemize}


	
\end{subsubsection} % un exemple simple fourni par leur tutorial
\end{subsection} % TFHE
\end{section}
