\begin{section}{Implémentation d'un FHE avec bootstrapping \og{} jouet\fg{}}
\begin{subsection}{Présentation de notre arborescence}
Nous avons concu une implémentation simple 
du cryptosystème GSW en sagemath, située dans le dossier
\path{GSW_implementation}. Celui-ci contient quatres dossiers:
\begin{itemize}
\item \path{GSW_scheme} contenant l'implémentation de GWS;
\item \path{analysis} contenant des fonctions permettant 
	de tester les fonctionnalités de notre implémentation 
	ou jncore de voir les performances 
	en terme de sécurités de certains choix de paramètres;
\item \path{unitary_test} contenant des tests assurant le bon fonctionnement
	des fonctions codées dans les autres dossiers; 
\item \path{lwe_estimator} contient les fichiers sources de l'API
	lwe\_estimator que nous avons présenté dans ce rapport; 
\end{itemize}

Nous proposons ici de faire une revue rapide des trois premiers dossiers. Notez que pour
\og attacher \fg avec sage un des sources, il faut rester à la racine pour
éviter des problèmes liés à l'utilisation de chemin relatifs pour les imports.


\begin{subsubsection}{GWS\_scheme}
Ce dossier contient les fichiers suivants:
\begin{itemize}
\item \path{GSW_scheme.sage} contient les fonctions principales de GWS
dont \path{setup}, \path{encrypt} et les 3 algorithmes de
déchiffrements que sont \path{basic_decrypt}, \path{mp_decrypt}
et \path{mp_all_q_decrypt}. Il contient aussi différentes variables
globales, dont \path{decrypt} permettant d'indiquer quel est
l'algorithme de dechiffrement par défault et les variables 
\path{bs_foo} indiquant quels paramètres sont utilisés lorsque 
on utilise des fonctins avec bootstrapping;
\item \path{auxilliary_functions.sage} contient l'implémentation 
	des diverses fonctions auxilliaires utilisées pour chiffrer 
	et déchiffrer les messages, comme par exemple \path{flatten};
	ou encore une implémentation du nearest plane de Babai;
\item \path{params_maker.sage} contient diverses fonctions permettant, à 
	partir d'un $n$, de retourner des paramètres $n, q, \chi, m$
	utilisés par le cryptosystème. Le fichier \path{GWS_scheme.sage}
	contient une variable globale \path{params_maker}
	permettant de fichier celui qu'utilise la fonction
	\path{sectup};
\item \path{homomorphic_functions.sage} contient la version homomorphe
	d'opérations de base comme la somme, ou encore le NAND;
\item \path{bootstrapping.sage} contient les fonctions nécessaires pour
	effectuer l'algorithme \path{basic_decrypt} homomorphiquement
	(il s'agit de la fonction \path{h_basic_decrypt}). On y trouve
	donc notamment diverses façon de sommer homomorphiquement
	des listes de chiffrés de $0$ et de $1$. Notez que la fonction
	\path{bootstrapping_arguments} permet d'effectuer un déchiffrement 
	homomorphique sur chacun des chiffrés
	passés en argument et que \path{setup_bs_params} permet d'initialiser 
	des variables globales \path{bs_pk, bs_sk, bs_lk, bs_params, bs_encrypted_sk}
	situées dans \path{bootstrapping.sage} et notamment utilisés dans
	\path{analysis/h_circuits_with_bootstrapping.sage}.
\end{itemize}
\end{subsubsection} % GSW_scheme

\begin{subsubsection}{analysis}
Ce dossier contient les fichiers suivants:
\begin{itemize}
\item \path{depth_security.sage} contenant l'implémentation de GWS;
\item \path{circuits.sage} contenant des fonctions permettant 
	de tester les fonctionnalités de notre implémentation 
	ou encore de voir les performances 
	en terme de sécurités de certains choix de paramètres;
\item \path{clear_functions.sage} contient des versions \og en clair \fg
	de fonctions homomorphes, utilisées dans les circuits;
\item \path{h_circuits_with_bootstrapping.sage} contient des exemples de
	fonctions utilisant des bootstrappings. Elles permettent de voir 
	si, pour certaines fonctions $f$, appliquer $f$ homomorphiquement 
	sur des chiffrés revient au même que d'abord l'appliquer sur les 
	clairs puis chiffrer.
	On peut toutes les lancer en  
	utilisant  la fonction \path{all_circuit_without_bs};
\item \path{h_circuits_without_bootstrapping.sage} contient des examples de
	fonctions n'utilisant pas de bootstrappings. 
	fonctions utilisant des bootstrappings. Elles permettent de voir 
	si, pour certaines fonctions $f$, appliquer $f$ homomorphiquement 
	sur des chiffrés revient au même que d'abord l'appliquer sur les 
	clairs puis chiffrer.
	On peut toutes les lancer en  
	utilisant  la fonction \path{all_circuit_with_bs};
\item \path{circuits.sage} Contient des fonctions permettant d'écrire 
	sous forme de string des fonctions algébriques simples, ce qui est utilisé
	dans  \path{h_circuits_without_bootstrapping.sage}. Par exemple, on 
	peut écrire \path{abc|*c+a~bc} pour signifier la fonction
	\[(a,b,c) \mapsto c * (a + (b \text{NAND} c)) \];
\item \path{all_circuit_analysis.sage} contient la fonction
	\path{analysis_main} qui lance les différents circuits avec et sans
	bootstrappings des fichiers précédents. 
\end{itemize}
\end{subsubsection} % analysis

\begin{subsubsection}{unitary\_test}
Ce dossier contient un fichier \path{framework_test.sage} permettant de mettre 
en forme les sorties des différentes fonctions de test, puis un fichier 
de test correspondant à chaque fichier du dossier \path{GSW_scheme}.
Chacun de ses fichiers contient une fonction \path{test_main_FOO}
ne demandant aucun argument et permettant de lancer les différents 
tests qu'il contient. De plus, le fichier \path{all_main_test.sage}
contient une fonction \path{test_main} permettant de lancer toutes 
les fonctions de forme \path{test_main_FOO}. On peut donc se faire une
idée du travail réalisé sur les tests en la lancant.
\begin{itemize}
\item \path{depth_security.sage} contenant l'implémentation de GWS;
\item \path{circuits.sage} contenant des fonctions permettant 
	de tester les fonctionnalités de notre implémentation 
	ou encore de voir les performances 
	en terme de sécurités de certains choix de paramètres;
\item \path{clear_functions.sage} contenant des tests assurant le bon fonctionnement
	des fonctions codées dans les autres dossiers; 
\item \path{h_circuits_with_bootstrapping.sage} contient les fichiers sources de l'API
	lwe\_estimator que nous avons présenté dans ce rapport; 
\item \path{h_circuits_without_bootstrapping.sage} contient les fichiers sources de l'API
	lwe\_estimator que nous avons présenté dans ce rapport; 
\end{itemize}
\end{subsubsection} % analysis

\end{subsection}
\end{section}
