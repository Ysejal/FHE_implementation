\begin{section}{Implémentation de GSW avec bootstrapping \og{}jouet\fg~{}}
\begin{subsection}{Présentation de notre arborescence}
Nous avons conçu une implémentation simple du cryptosystème GSW en sagemath, située dans le dossier
\path{GSW_implementation}. Celui-ci contient les  dossiers suivants:
\begin{itemize}
\item \path{GSW_scheme} contenant l'implémentation du cryptosystème GSW;
\item \path{analysis} contenant des fonctions permettant 
	de tester notre implémentation sur divers circuits homomorphes ainsi que de voir les performances en terme
	de sécurités et de profondeur de NAND possible de certains choix de paramètres;
\item \path{unitary_test} contenant des tests assurant le bon fonctionnement
	des fonctions codées dans les autres dossiers; 
\item \path{lwe_estimator} contenant les fichiers sources de l'API
	 lwe\textunderscore estimator présentée dans ce rapport dans la section~\ref{estimator} page \pageref{estimator}; 
\end{itemize}

Nous proposons ici de faire une revue rapide des trois premiers dossiers. Notez que pour
attacher avec sage un des fichiers sources, il faut rester au dossier racine pour
éviter des problèmes liés à l'utilisation de chemin relatifs pour les imports.


\begin{subsubsection}{GWS\_scheme}
Ce dossier contient les fichiers suivants:
\begin{itemize}
\item \path{GSW_scheme.sage} contient les fonctions principales de GWS dont \path{setup}, \path{encrypt}, \path{keys_gen} et les 3
algorithmes de déchiffrements que sont \path{basic_decrypt} (l'algorithme \textbf{Decrypt} du rapport), \path{mp_decrypt}
et \path{mp_all_q_decrypt}. Il contient aussi différentes variables globales, dont \path{decrypt} permettant d'indiquer
quel est l'algorithme de déchiffrement par défaut;
\item \path{auxilliary_functions.sage} contient l'implémentation des diverses fonctions auxiliaires utilisées pour
chiffrer et déchiffrer les messages, comme par exemple \path{flatten}, ou encore une implémentation du \textbf{nearest plane} de
Babai;
\item \path{params_maker.sage} contient diverses fonctions permettant, à partir d'un $n$, de retourner des paramètres
$n, q, \chi, m$ utilisés par le cryptosystème. Le fichier \path{GWS_scheme.sage} contient alors une variable globale
\path{params_maker} permettant de choisir lequel utilise la fonction \path{setup} du cryptosystème GSW; 
\item \path{homomorphic_functions.sage} contient la version homomorphe
	d'opérations de base comme la somme ou encore le NAND;
\item \path{bootstrapping.sage} contient les fonctions nécessaires pour
	effectuer l'algorithme \path{basic_decrypt} homomorphiquement (il s'agit de la fonction \path{h_basic_decrypt}).
	On y trouve donc notamment les diverses façon de sommer homomorphiquement des listes de chiffrés de 0 et de
	1 présentées dans la sous-section~\ref{sec:sum_lists} page \ref{sec:sum_lists}. Notez que la fonction
	\path{bootstrapping_arguments} permet d'effectuer un déchiffrement homomorphe sur chacun des chiffrés passés
	en argument et que \path{setup_bs_params} permet d'initialiser des variables globales \path{bs_pk, bs_sk, bs_lk,
	bs_params, bs_encrypted_sk} situées dans \path{bootstrapping.sage} et notamment utilisés dans
	\path{analysis/h_circuits_with_bootstrapping.sage}. 
	\end{itemize}
\end{subsubsection} % GSW_scheme

\begin{subsubsection}{analysis}
Ce dossier contient les fichiers suivants:
\begin{itemize}
\item \path{depth_security.sage} contient une fonction permettant de lancer \path{lwe_estimator} sur des paramètres définis dans \path{params_maker.sage} et une autre permettant de voir quel est le plus grand $L$ satisfaisant la condition de longueur (équation \eqref{eq:longueur} page \pageref{eq:longueur}) sur des paramètres définis dans \path{params_maker.sage}
\item \path{h_circuits_with_bootstrapping.sage} contient des exemples de fonctions utilisant le bootstrapping. Elles
permettent de voir si, pour certaines fonctions $f$, appliquer $f$ homomorphiquement sur des chiffrés permet bien d'obtenir un chiffré de l'application de $f$ aux clairs.  On peut toutes les lancer en  utilisant  la fonction
\path{all_circuit_with_bs};
\item \path{h_circuits_without_bootstrapping.sage} contient des examples de fonctions n'utilisant pas de bootstrappings.
Tout comme les fonctions de \path{h_circuits_with_bootstrapping.sage}, elles permettent de voir si, pour certaines fonctions $f$, appliquer $f$
homomorphiquement sur des chiffrés permet bien d'obtenir un chiffré de l'application de $f$ aux clairs.  On peut toutes les lancer en  utilisant  la fonction \path{all_circuit_without_bs};
\item \path{all_circuit_analysis.sage} contient la fonction \path{analysis_main} qui lance les différentes fonctions avec
et sans bootstrappings des deux fichiers précédents;
\item \path{circuits.sage} contient des fonctions permettant d'écrire sous forme de string des fonctions algébriques
simples, ce qui est utilisé dans  \path{h_circuits_without_bootstrapping.sage}. Par exemple, on peut écrire
\path{abc|*c+a~bc} pour signifier la fonction \[(a,b,c) \mapsto c * (a + (b\:\: \text{NAND}\:\: c)) \]
\end{itemize}
\end{subsubsection} % analysis

\begin{subsubsection}{unitary\_test}
Ce dossier contient un fichier \path{framework_test.sage} permettant de mettre en forme les sorties des différentes
fonctions de test, ainsi qu'un fichier de test correspondant à chaque fichier du dossier \path{GSW_scheme} permettant de s'assurer du bon fonctionnement des fonctions du fichier associé.  Chacun de ces
fichiers contient une fonction \path{test_main_FOO} ne demandant aucun argument et permettant de lancer les différents
tests qu'il contient. De plus, le fichier \path{all_main_test.sage} contient une fonction \path{test_main} permettant de
lancer toutes les fonctions de forme \path{test_main_FOO} des autres fichiers de test.  On peut donc se faire une idée
du travail réalisé sur les tests en la lançant.
\end{subsubsection} % unitary_test
\end{subsection}
\end{section}
