\begin{section}{Mise en place d'un bootstrapping}
\begin{subsection}{Un point sur la sécurité}
JE N'AI PAS RETROUVÉ LE TERME UTILISÉ PAR MR. CASTAGNOS SUR LA SÉCURITÉ ICI.
CIRCULAIRE ? CYCLIQUE ? UN TRUC DE CE GENRE.\\
Nous avons vu que le système cryptographique que nous étudions est IND-CPA, ce qui est le niveau de sécurité théorique que l'on veut généralement. La première question qui se pose pour le bootstrapping est de savoir si l'on garde ce niveau de sécurité.\\
Le problème est que pour effectuer un déchiffrement homomorphe, il faut au moins un chiffré de la première clé secrète par la deuxième clé publique. \\
Il faut pouvoir s'assurer qu'un attaquant ne puisse pas en tirer d'information sur la première clé. Cependant, le système étant IND-CPA, cette propriété est toujours vérifié lorsque les deux clés sont indépendantes l'une de l'autre. \\
Il suffit donc de générer les clés indépendamment les unes des autres pour s'assurer d'avoir le niveau de sécurité désiré.
\end{subsection}
\begin{subsection}{Un premier découpage}
Afin de pouvoir effectuer un bootstrapping à partir de l'algorithme de déchiffrement \textbf{Dec}, nous allons avoir besoin de l'exprimer uniquement à partir d'opérations \textbf{NAND} sur des 0 et des 1. \\
N'étant pas sous la contrainte d'être particulièrement performant, nous avons écris des fonctions booléennes homomorphiques \path{NO}, \path{AND}, \path{OR} et \path{XOR} avec des portes \textbf{NAND}. \\ DETAILS NECESSAIRES (nb NAND, profondeur) \\
Cela nous permettra de l'exécuter homomorphiquement. \\
Pour cela, nous considèrerons la décomposition binaire du secret et considèreront avoir un chiffré par la seconde clé de chaque bit du premier secret.
\end{subsection}
\begin{subsection}{Sommer des vecteurs avec une profondeur en NAND minimale}
Nous voulons pouvoir sommer homomorphiquement des nombres à partir des chiffrés de leurs décompositions binaires traitées comme des listes de tailles $lim$. Pour cela, nous allons étudier deux algorithmes. \\

\paragraph{}

Le premier, que nous appelerons \textbf{basic\_sum} est l'algorithme naïf de somme binaire, commençant par les bits de poids faible puis remontant vers les bits de poids plus élevés en conservant des retenues. \\
Soient $A = \sum\limits_{i=0}^{lim} a_i 2^i$ et $B = \sum\limits_{i=0}^{lim} b_i 2^i$ que l'ont veux sommer. \\
On nomme $s_0, s_1, ..., s_{lim}$ les bits de la somme et $r_i$ la retenue après être passé sur les bits $a_i$ et $b_i$. \\
On définit $r_{-1} = 0$. \\
On calcule alors juste : \\
$s_i = a_i \oplus b_i \oplus r_{i-1}$ et \\
$r_i = (a_i \land b_i) \lor (r_{i-1} \land (a_i \lor b_i))$. \\
Le problème de cette méthode est que l'on réutilise la $r_{i-1}$ pour calculer $r_i$. Cela signifie entre autre que $a_0$ et $b_0$ sont utilisés tous les calculs. \\
Calculer $s_i$ peut se faire en utilisant $r_{i-1}$ que pour un $\oplus$, ce qui n'ajoute que 4 à la profondeur en \textbf{NAND} du calcul. \\
$r_i$ peut être calculé en appliquant un \path{AND} et un \path{OR} à $r_{i-1}$, ce qui ajoute aussi 4 à la profondeur. \\
$r_0 = a_0 \land b_0$ et peut donc être trouvé avec une profondeur de 2 \textbf{NAND}. \\
$A + B$ se calcule donc avec une profondeur de $4 lim + 2$ \textbf{NAND} en $a_0$ et $b_0$.

\paragraph{}

Le second, que nous appelerons \textbf{reduced\_sum} est un algorithme assez utilisé qui à trois nombres écris en binaire associe deux listes de tailles $lim$ dont la somme est égale à la somme des trois nombres donnés en entrée. \\
Soient $A = \sum\limits_{i=0}^{lim} a_i 2^i$, $B = \sum\limits_{i=0}^{lim} b_i 2^i$ et $C = \sum\limits_{i=0}^{lim} c_i 2^i$ que l'ont veux sommer. \\
On nomme $X = \sum\limits_{i=0}^{lim} x_i 2^i$  et $Y = \sum\limits_{i=0}^{lim} y_i 2^i$ les nombres obtenus. \\
On fixe $y_0 = 0$. \\
On calcule ensuite : \\
$x_i = a_i \oplus b_i \oplus c_i$ et \\
$y_i = \overline{(\overline{a_{i-1}} \land \overline{b_{i-1}}) \oplus (\overline{b_{i-1}} \land \overline{c_{i-1}}) \oplus (\overline{a_{i-1}} \land \overline{c_{i-1}})}$ pour $i \neq 0$. \\
On remarque que cette méthode, contrairement à \textbf{basic\_sum}, n'effectue pas de récurence. La profondeur totale sera donc le max de la profondeur du calcul de $x_i$ et de celle du calcul de $y_i$. \\
Calculer $x_i$ consiste en deux $\oplus$ succesifs dont le deuxième utilise le résultat du premier. La profondeur en \textbf{NAND} est donc de 8. \\
La profondeur maximale du calcul de $y_i$ est celle des éléments impliqués dans deux $\oplus$. Ces éléments subissent donc deux \path{NO}, un \path{AND} et deux \path{XOR}, atteignant ainsi une profondeur de 12 \textbf{NAND}.

\end{subsection}
\begin{subsection}{Prendre la valeur absolue dans $\ZZq$}
\end{subsection}
\end{section}
